\documentclass[pdftex,aspectratio=169]{beamer}

% Setup beamer look theme
\mode<presentation>
{
	\useinnertheme{rectangles}
	\useoutertheme{infolines}
	\usecolortheme{crane}
	\setbeamertemplate{navigation symbols}{}
}

% Use nicer font
\usepackage{palatino}
\usepackage{palatino,graphics,array}

% Setup UTF-8 encoding
\usepackage[utf8]{inputenc}
\usepackage[T1]{fontenc}

% Setup czech language
\usepackage[czech]{babel}

% Include usefull packages
\usepackage{graphics}

% Include hyperlinks support
\usepackage{hyperref}
\hypersetup{colorlinks=true,linkcolor=black,urlcolor=black,unicode=true}

% Add support for simple URLs
\usepackage{url}

% Commands for CC license
\newcommand{\CcLongnameByNcSa}{\href{http://creativecommons.org/licenses/by-nc-sa/3.0/cz/}{Attribution-Noncommercial-ShareAlike}}
\newcommand{\CcImageBy}[1]{\includegraphics[scale=#1]{../commons/graphics/cc-by-white.pdf}}
\newcommand{\CcImageNc}[1]{\includegraphics[scale=#1]{../commons/graphics/cc-nc-white.pdf}}
\newcommand{\CcImageSa}[1]{\includegraphics[scale=#1]{../commons/graphics/cc-sa-white.pdf}}
\newcommand{\CcGroupByNcSa}[2]{\CcImageBy{#1}\hspace*{#2}\CcImageNc{#1}\hspace*{#2}\CcImageSa{#1}}




\title{PB173 Tématický vývoj aplikací v C/C++}
\subtitle{Software pro obsluhu výpočtů v gridovém prostředí}
\author[]{Mgr.~Šimon~Tóth}
\institute[CESNET]{Cesnet z.s.p.o.}
\date{\today}

\newcommand{\CcNote}[1]{% longname
        Licencováno pod: \textit{Creative Commons #1 3.0 License}.%
}

\begin{document}
	\begin{frame}
		\titlepage
		\vfill
		\begin{center}
			\CcGroupByNcSa{0.33}{0.95ex}\\
			{\tiny\CcNote{\CcLongnameByNcSa}}
			\vspace*{2ex}
		\end{center}
	\end{frame}

\section{Organizační info}
\subsection{Představení}
        
\begin{frame}
	\frametitle{Představení}
	\begin{itemize}
		\item{Mgr. Šimon Tóth}
		\begin{itemize}
			\item{externí lektor C a C++ na FI}
			\item{výzkumný pracovník Cesnet}
			\item{Ph.D. student}
			\item{kontaktní info na konci slidů}
		\end{itemize}
	\end{itemize}
\end{frame}

\subsection{Průběh předmětu}

\begin{frame}
	\frametitle{Průběh předmětu}
	\begin{itemize}
		\item{pouze cvičení}
		\item{ukončení kolokviem}
		\item{průběžně domácí úkoly - převážně dokončení práce z cvičení}
		\item{závěrečný projekt - ve skupinkách}
	\end{itemize}
\end{frame}

\subsection{Obsahové zaměření}

\begin{frame}
	\frametitle{O čem to určitě nebude}
	\begin{itemize}
		\item{knihovny pro paralelní výpočty}
		\item{high performance computing}
		\item{implementační detaily výpočetních úloh}
	\end{itemize}
\end{frame}

\begin{frame}
	\frametitle{Co od toho očekávate?}
	\begin{itemize}
		\item{proč jste si tuto skupinu zapsali?}
		\item{co by jste se chtěli naučit?}
	\end{itemize}
\end{frame}

\begin{frame}
	\frametitle{O čem to může být}
	\begin{itemize}
		\item{open-source software a práce s upstreamem}
		\item{práce s existujícími projekty s dlouhou historií}
		\item{dostupné nástroje}
		\item{práce s modelem server-klient}
		\item{udržování kritického (infrastrukturního) SW}
		\item{asynchronní vs. synchronní model}
	\end{itemize}
\end{frame}

\begin{frame}
	\frametitle{Možnosti uplatnění}
	\begin{itemize}
		\item{MetaCentrum}
		\item{bakalářské a diplomové práce}
		\item{stipendia a částečné úvazky}
	\end{itemize}
\end{frame}

\subsection{Požadavky}

\begin{frame}
	\frametitle{Vstupní požadavky}
	\begin{itemize}
		\item{práce s UNIXem, LINUXem na uživatelské úrovni}
		\item{znalost jazyka C na slušné úrovni}
		\item{orientace v UNIXové dokumentaci}
		\item{možná narazíme i na C++}
	\end{itemize}
\end{frame}

\subsection{Materiály}

\begin{frame}
	\frametitle{Výukové materiály}
	\begin{itemize}
		\item{\url{https://minotaur.fi.muni.cz:8443/~xsvenda/docuwiki/doku.php?id=public:pb173:pb173\_2011\_grid}}
		\item{\url{https://github.com/HappyCerberus/teaching}}
		\item{\url{http://stackoverflow.com/}}
	\end{itemize}
\end{frame}

\section{Představení projektů}
\subsection{CESNET}

\begin{frame}
	\frametitle{CESNET}
	\begin{itemize}
		\item{správce České Národní Akademické Sítě}
		\item{služby poskytuje univerzitám a výzkumným organizacím}
	\end{itemize}
\end{frame}

\begin{frame}
	\frametitle{MetaCentrum}
	\begin{itemize}
		\item{správce České Národní Gridové Infrastruktury}
		\item{služby dostupné pro libovolný nekomerční výzkum}
	\end{itemize}
\end{frame}

\subsection{Torque}

\begin{frame}
	\frametitle{Torque}
	\begin{itemize}
		\item{batch systém}
		\item{více než dvacetiletá historie}
		\item{původně OpenPBS -> fork do Torque a PBS Pro}
	\end{itemize}
\end{frame}

\section{Praktická část cvičení}
\subsection{Přehled}

\begin{frame}
	\frametitle{Dnešní praktická část}
	\begin{itemize}
		\item{stažení kódu}
		\item{orientace v kódu}
		\item{git, doxygen, eclipse}
	\end{itemize}
\end{frame}

\subsection{Forkování projektu}

\begin{frame}
	\frametitle{Forkování SVN}
	\begin{itemize}
		\item{SVN je centralizovaný VCS}
		\item{pro commit je potřeba přístup na server}
		\item{pro údržbu lokálního forku potřebujeme commitovat}
	\end{itemize}
\end{frame}

\begin{frame}
	\frametitle{GIT}
	\begin{itemize}
		\item{\url{http://git-scm.com}}
		\item{\texttt{git svn clone --stdlayout URL}}
		\item{ukázka práce s GITem}
		\item{\url{http://files.simontoth.cz/torque.tbz}}
	\end{itemize}
\end{frame}

\subsection{Orientace v kódu}

\begin{frame}
	\frametitle{Doxygen}
	\begin{itemize}
		\item{ukázka výstupu doxygenu pro Torque}
		\item{\url{http://files.simontoth.cz/doxydoc.tbz}
	\end{itemize}
\end{frame}

\begin{frame}
	\frametitle{Eclipse}
	\begin{itemize}
		\item{ukázka práce s Eclipse}
		\item{callgraphy pro funkce a atributy}
	\end{itemize}
\end{frame}

\section{Kontaktní informace}
	\subsection{Pracovní}

\begin{frame}[label=kontakt-simontoth]
	\frametitle{Mgr. Šimon Tóth}
	\begin{itemize}
		\item{Gotex (Šumavská 15) - kanceláře Cesnet (B310)}
	\end{itemize}

	\begin{columns}
	\column{0.5\textwidth}
		Fakulta Informatiky (MU)
		\begin{itemize}
			\item{toth@fi.muni.cz}
			\item{tel. 549 49 6446}
		\end{itemize}

	\column{0.5\textwidth}
		Cesnet z.s.p.o.
		\begin{itemize}
			\item{simon@cesnet.cz}
			\item{kancelář C122}
			\item{tel. 234 680 235}
		\end{itemize}

	\end{columns}
\end{frame}

	\subsection{Osobní}

\begin{frame}
	\frametitle{Mgr. Šimon Tóth}
		Osobní
		\begin{itemize}
			\item{kontakt@simontoth.cz}
			\item{tel. 776 565 424}
		\end{itemize}
\end{frame}




\end{document}
