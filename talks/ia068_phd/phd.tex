\documentclass[pdftex,aspectratio=169]{beamer}

% Setup beamer look theme
\mode<presentation>
{
	\useinnertheme{rectangles}
	\useoutertheme{infolines}
	\usecolortheme{crane}
	\setbeamertemplate{navigation symbols}{}
}

% Use nicer font
\usepackage{palatino}
\usepackage{palatino,graphics,array}

% Setup UTF-8 encoding
\usepackage[utf8]{inputenc}
\usepackage[T1]{fontenc}

% Setup czech language
\usepackage[czech]{babel}

% Include usefull packages
\usepackage{graphics}

% Include hyperlinks support
\usepackage{hyperref}
\hypersetup{colorlinks=true,linkcolor=black,urlcolor=black,unicode=true}

% Add support for simple URLs
\usepackage{url}

% Commands for CC license
\newcommand{\CcLongnameByNcSa}{\href{http://creativecommons.org/licenses/by-nc-sa/3.0/cz/}{Attribution-Noncommercial-ShareAlike}}
\newcommand{\CcImageBy}[1]{\includegraphics[scale=#1]{../commons/graphics/cc-by-white.pdf}}
\newcommand{\CcImageNc}[1]{\includegraphics[scale=#1]{../commons/graphics/cc-nc-white.pdf}}
\newcommand{\CcImageSa}[1]{\includegraphics[scale=#1]{../commons/graphics/cc-sa-white.pdf}}
\newcommand{\CcGroupByNcSa}[2]{\CcImageBy{#1}\hspace*{#2}\CcImageNc{#1}\hspace*{#2}\CcImageSa{#1}}




\title{Scheduling the GRID(s)}
\subtitle{an introduction}
\author[]{Mgr.~Šimon~Tóth}
\institute[FI@MU]{Fakulta informatiky @ Masarykova Univerzita}
\date{\today}

\newcommand{\CcNote}[1]{% longname
        Licencováno pod: \textit{Creative Commons #1 3.0 License}.%
}

\begin{document}

\pgfdeclarelayer{background}
\pgfsetlayers{background,main}

\begin{frame}
\titlepage
	\vfill
	\begin{center}
		\CcGroupByNcSa{0.33}{0.95ex}\\ {\tiny\CcNote{\CcLongnameByNcSa}}
		\vspace*{2ex}
	\end{center}
\end{frame}

% Thank you all for comming here. My name is Simon Toth and I would like to
% introduce you today into the area of grid scheduling, with particular
% emphasis on distributed scheduling.

% Because this is my first year of PhD studies, I will be mostly trying to
% introduce you into the area of grid scheduling and the challenges we are
% facing.

{
\usebackgroundtemplate{\includegraphics[width=\paperwidth]{imgs/ibm-blue-gene.jpg}}
\frame{}
}

{
\usebackgroundtemplate{\includegraphics[width=\paperwidth]{imgs/ibm-blue-gene-back.jpg}}
\begin{frame}

	\begin{columns}
	\column{0.45\textwidth}
	\begin{block}{Hardware models}
	\begin{overlayarea}{\textwidth}{4cm}
		\begin{itemize}
			\item<1->{super-computer}
			\item<2->{computer cluster}
			\item<3->{computational grid}
			\item<4->{scientific cloud}
		\end{itemize}
		\vspace{5cm}
	\end{overlayarea}
	\end{block}

	\column{0.45\textwidth}
	\begin{block}{Features}
	\begin{overlayarea}{\textwidth}{4cm}
	\only<1>{
	\small
	\begin{itemize}
		\item maximum performance
		\item not suitable for generic computing
		\item expensive
		\item specialized software
	\end{itemize}
	}
	\only<2>{
	\small
	\begin{itemize}
		\item maximum usability
		\item homogenous environemnt
		\item weighted performance/cost
	\end{itemize}
	}
	\only<3>{
	\small
	\begin{itemize}
		\item maximum utility
		\item heterogenous environment
		\item weighted usage/cost
	\end{itemize}
	}
	\only<4>{
	\small
	\begin{itemize}
		\item maximum gain
		\item dynamic environment
		\item weighted cpu time/cost
	\end{itemize}
	}
	\vspace{5cm}
	\end{overlayarea}
	\end{block}
	\end{columns}

\end{frame}

\begin{frame}

\end{frame}
}


\begin{frame}
	\frametitle{Cloud}

	cloud computing
	CPU time billing model
	high flexibility, dynamic infrastructure
	immediate scalability
\end{frame}


\section{Scheduling on grids}

\begin{frame}
	\frametitle{Scheduling on Grids}
	ukazka planovani jobu pro minimalizaci slowdown - dynamicke programovani
	vysvetleni problemu non-clairvoirance
	planovani bez zname delky jobu
	vysvetleni problemu on-line scheduling
\end{frame}

% lack of criteria
%	non-comparability of solutions

\section{Current approaches}
\subsection{FIFO based solutions}
\subsection{Schedule based solutions}

% FIFO based solutions
%	jobs are processed on a first-come first-serve basis
%	mutliple extensions using priorities, jobs starvations, etc...
%	easy to grasp, hard to manage (each configuration option affects all other configuration options)
%	can be very britle

% Schedule based solutions
%	the scheduler creates a real job schedule
%	scalability and performance issues
%	problems with very dynamic configurations

% Scalability and stability
% 

% Approaching the distributed environment
%	bin packing

% Deadline scheduling

% One of the possible approaches that could allow us to escape the current rigid model is deadline scheduling.
% In deadline scheduling, the scheduler is trying to create a schedule that will violate the least amount of deadlines.
% We are depending largely on the information provided by the oracle, since we don't have any other source of precise job runtime.

\end{document}
