\documentclass[12pt,a4paper]{article}

% Setup UTF-8 encoding
\usepackage[utf8]{inputenc}
\usepackage[T1]{fontenc}

% Setup czech language
\usepackage[english]{babel}

\usepackage{a4wide}

\begin{document}

\section{Introduction}

Welcome, thank you all for comming here today. My name is Šimon Tóth and I
would like to introduce you into the area of job scheduling in GRIDs.  My main
job is work for CESNET, where I work in MetaCentrum, the Czech National Grid
Infrastructure maintainer.  My Ph.D. studies are therefore concentrating on
this area and are conducted in cooperation with both MetaCentrum and CERIT
(which is the new GRID and cloud provider founded by the University).

In the next 20 minutes I will try to present to you the chalenges and problems
associted with job scheduling.  If at any time you feel like there is something
that should be clarified, or you have any question, don't hesitate and ask.

\subsection{Quiz}

\texttt{\textbf{---SWITCH SLIDE---}}

Just to get a quick overview of how many people here are familiar with this
subject, allow me a short quiz.

Please raise your hand:

\begin{itemize}
	\item if you are familiar with job scheduling
	\item if you know what a grid is
	\item if you ever heard about MetaCentrum or CERIT
	\item if you ever used a grid
\end{itemize}

\section{Computational Resources}

Those of you who know what a grid is may have some visual image of the grid in
your mind.  If the image is something like the one you see on the slide, than
you are actually wrong, because this isn't a grid.  This is one of the older
version of BlueGene, which is a super-computer.

While the difference between a super computer and a grid might not seem
significant, allow me please to go more deeper into this issue because the
difference between different computational resources are actually very
important.

\subsection{super-computer}

\texttt{\textbf{---SWITCH SLIDE---}}

Lets start with supercomputers.

While historically computers started as super-computers (huge machines, built
for maximum performance with no, or minimal regard towards price).  We have
quickly shifted towards cheaper and more practical alternatives.

At this time, you would want a super computer if either you want maximum
possible performance, or when you have a use case for which you have some
estimate of required processing power.

Supercomputers are still the most expensive computational model, and therefore
you will be usually buying a supercomputer for a long period of time. Most
supercomputers are not designed for generic computation, but are usually optimized for some specific use case.

\subsection{clusters}

\texttt{\textbf{---SWITCH SLIDE---}}

Another interesting model is computer cluster.
Computer clusters are of course great for use cases where the workload can be distributed onto multiple machines.
Another charm of clusters is of course that you can get reasonable performance for much less money than a super computer would cost.

One very specific feature of a computer cluster is that it is homogenous.
You have identical machines, connected together with a high speed network.
Now because of this, it is quite cheeap to upgrade a cluster, you can simply buy more identical machines.
But it has one big weakness. If you are buying identical machines, you are concentrating on one use case.
If you buy generic machines, you are building generic performance, if you buy machines with GPU cards, you are building a cluster suitable for GPU computing.
The problem arises when you have diverse users. Users that want CPU performance, won't get much of that from a GPU cluster.

This is where grids come into play.

\subsection{grid}

\texttt{\textbf{---SWITCH SLIDE---}}

Grids are a congregation of resource providers, that are operating clusters and are willing to agree upon some level of infrastructural commonality to allow users easy access to these clusters.
This can range from "simple" things like shared user database to complex solutions include middle-ware and even OS.

Czech National Grid is a very tightly integrated grid, where the shared infrastructure goes even to the OS level, but we are still capable to support sites that want to maintain some level of independence.

A grid environment solves the issues of specific user requirements by facilitating the users to access clusters of other organizations, which may have different attributes.

\subsection{cloud}

Of course I can't skip clouds. Clouds present a slightly strange model, because Cloud is actually a business model. It a first really working business model for selling computing time.
I won't be covering cloud in this talk, but since the Czech National Grid has a modern virtualized infrastructure, I will mention a few cloud-like aspects.

\subsection{grids again}

So these are the computing hardware model. Of course I just presented a very rough simplification, but hopefully, you can now imagige where the GRID fits in.

Let me come back to grid for a short moment and introduce a more notions that I will be using during this presentation.
In a grid environment, users are represented by virtual ogranizations.

Virtual organizations negotiate access to resources with resource providers. This can include a fully fledged contract between the resource provider and the virtual organization.

The entity that mostly enters as technology and know-how provider I will be refering to as grid maintainer/operator.

\section{Metaphor}

Now that we know the playfield, lets move on to job scheduling.
And I would like to introduce job scheduling in the GRID context using a metaphor.

\subsection{Marketplace}

Imagine a marketplace. On a normal marketplace, customers enter and leave
without any control. The problem with this model is that when a customer enters
a marketplace, he has no guarantee that he will get what he wants.

The market place may not be able to provide the goods he is searching for at all, or they may be sold out.

Now imagine that you want to improve this situation.
You would want to provide a service that would tell the customers whether the goods that their require are currently still present in the marketplace.
For that you would need some sort of central authority and only one guarded entrace to the marketplace.

\subsection{Mapping}

How does this metaphor map to GRID scheduling? Well, the original marketplace
with no control at all is like buying computers and giving users shell access
to these machines.

Now of course, users are greedy and if you use this approach, very soon you
will run into issues of users using the entire capacity of the machines for
themselves.

The other problem you will run into are guarantees of resources. If a computer
program runs out of memory, it will either crash, or end in some other
forcefull manner. Since many users need to run long (even month long)
computations, you really need to provide some sort of guarantee that the
requested amount of memory will be available on the machine through the entire
life time of the computation.

\section{Central authority}

So what does the central authority have to provide?

First of all, we really need to force all users to enter only through this
central authority. So while we may allow users shell access to machines, we
definitelly won't allow any computational jobs being run manually on the machines.

The central authority has to provide the guarantees of resources, therefore we
need to do static allocations. This also allows the central authority to have
less current information about the GRID, because the central authority can
easilly track the current state of each machine locally.

Of course, machine failures still have to be monitored.

\subsubsection{Job scheduling}

So, how hard this problem actually is? Well let's start with the simple variant.
You have some jobs, each with some different length.  And to keep it simple, we
want all of these jobs to finish as soon as possible.  

Example of offline job scheduling.

Unfortunatelly for us, the GRID scheduling problem isn't as simple as that.

One of the problematic parts of GRID scheduling is the combinatoric explosion.
In theoretical problems and solutions, the problem is extremely oversimplified.

First, jobs can be very picky about the machine they want to run on.
So for example you could be only able to run job1, job2, and job3 on machine1 or machine2

In our example the jobs had only one dimension, the length.  In actuallity GRID
jobs are 3 or more dimensional. Each job has length, number of CPUs and memory
requirement. Some jobs have GPU requirements, some have license requirements,
some have disk space requirements.

While the number of dimensions will increase the complexity of the algorith, it
will only result in higher computational complexity. This is a big problem, but
actually the smallest.

One of the big issues is non-claivoirance. We don't really know how long the
jobs will be.  We have some upper limit. But a job with a 24 hour limit can
easily end after 15 minutes.

Problem s online characteristikou problemu.

Problem s merenim uspesnosti rozvrhu.

- how hard is this problem
  - job scheduling
  - oraculum
  - non-claivoirance
  - on-line problem

\subsection{Measuring the success}

What do you do when you have an algorithm that can actually provide a schedule.
How do you compare different algorithms?

What is a good schedule. Now from a user perspective, the only measurement is
how fast the computational jobs get computed. Its as simple as that.

From the resource provider perspective, this is a bit more complex. 

\subsection{Big marketplace}

Lets come back to the marketplace metaphor. So you are now aware of the issue
connected with job sheduling on a single grid aka a marketplace.  Now what
happens, when the marketplace gets to big to maintain. The only option is to split it.

Since the bottle-neck is the central entrance point, we basically want to build
new walls and new entrances into the market place.

The good thing is that we are still in control of the entire market place, and
it is us, who set the policies.

\subsection{Another marketplace}

But this model fails, when we introduce another marketplace into the play.

With one market place being split into multiple parts we still got the controll about the policies we enforce.
With another market place with full autonomy, we can only rely on the agreements established between these two market places.

\end{document}
