\documentclass[pdftex,aspectratio=169]{beamer}

% Setup beamer look theme
\mode<presentation>
{
	\useinnertheme{rectangles}
	\useoutertheme{infolines}
	\usecolortheme{crane}
	\setbeamertemplate{navigation symbols}{}
}

% Use nicer font
\usepackage{palatino}
\usepackage{palatino,graphics,array}

% Setup UTF-8 encoding
\usepackage[utf8]{inputenc}
\usepackage[T1]{fontenc}

% Setup czech language
\usepackage[czech]{babel}

% Include usefull packages
\usepackage{graphics}

% Include hyperlinks support
\usepackage{hyperref}
\hypersetup{colorlinks=true,linkcolor=black,urlcolor=black,unicode=true}

% Add support for simple URLs
\usepackage{url}

% Commands for CC license
\newcommand{\CcLongnameByNcSa}{\href{http://creativecommons.org/licenses/by-nc-sa/3.0/cz/}{Attribution-Noncommercial-ShareAlike}}
\newcommand{\CcImageBy}[1]{\includegraphics[scale=#1]{../commons/graphics/cc-by-white.pdf}}
\newcommand{\CcImageNc}[1]{\includegraphics[scale=#1]{../commons/graphics/cc-nc-white.pdf}}
\newcommand{\CcImageSa}[1]{\includegraphics[scale=#1]{../commons/graphics/cc-sa-white.pdf}}
\newcommand{\CcGroupByNcSa}[2]{\CcImageBy{#1}\hspace*{#2}\CcImageNc{#1}\hspace*{#2}\CcImageSa{#1}}




\title{Návaznost jazyka C na OS}
\subtitle{Přednášeno v rámci předmětu PB071}
\author[]{Mgr.~Šimon~Tóth}
\institute[FI@MU]{Fakulta informatiky @ Masarykova Univerzita}
\date{\today}

\newcommand{\CcNote}[1]{% longname
        Licencováno pod: \textit{Creative Commons #1 3.0 License}.%
}

\begin{document}

\begin{frame}
  \titlepage
\end{frame}

\begin{frame}
  \frametitle{Návaznost jazyka C na OS}
  \tableofcontents
\end{frame}

\section{Historické návaznosti}
\subsection{Programovací jazyky a jejich vztah k OS}

\begin{frame}
	\frametitle{Přístup programovacích jazyků k multi-platformnosti}

\begin{columns}
	\column{.45\textwidth}
	\begin{block}<1->{Multiplatformní framework}
	\scriptsize
		\begin{itemize}
			\item{framework zastřešuje práci s OS}
			\item{aplikační kód staví na frameworku}
			\item{Jazyky: Java, .Net/Mono}
		\end{itemize}
	\end{block}
	\begin{block}<4->{Multiplatformní jazyk}
	\scriptsize
		\begin{itemize}
			\item{multiplatformnost přímo v návrhu jazyka}
			\item{velmi malé množství předpokladů}
			\item{vazby na OS přenechány knihovnám}
			\item{Jazyky: C, C++}
		\end{itemize}
	\end{block}
	\column{.45\textwidth}
	\begin{block}<2->{Specializované jazyky}
	\scriptsize
		\begin{itemize}
			\item{specializované využití}
			\item{multiplatformnost buďto nemá smysl uvažovat}
			\item{Jazyky: AWK, Javascript}
		\end{itemize}
	\end{block}
	\begin{block}<3->{Skriptovací jazyky}
	\scriptsize
		\begin{itemize}
			\item{program se vykonává skrz interpret/virtuální stroj}
			\item{multiplatformní v závislosti na interpretu}
			\item{vazby na OS jak specifické tak obecné}
			\item{Jazyky: Perl, Python, PHP}
		\end{itemize}
	\end{block}
\end{columns}

\end{frame}

\subsection{Unix a Windows}
\begin{frame}
	\frametitle{Unix vs. Windows}
	\begin{itemize}
		\item{kořeny Windows začínají u jazyka Basic}
		\item{C podporováno v Microsoft C od roku 1983}
		\item{rychle přebito jazykem C++}\pause
		\begin{itemize}
			\item{první podpora C 6.0 (1989)}
			\item{plnohodnotná podpora C/C++ 7.0 (1992)}
			\item{Visual C++ (1993)}
		\end{itemize}\pause
		\item{pro porovnání}
		\begin{itemize}
			\item{ANSI C 1989 / ISO C 1990}
			\item{C99 (1999)}
			\item{ANSI C++ 1998 / revize 2003}
		\end{itemize}
	\end{itemize}
\end{frame}

\begin{frame}
	\frametitle{Aktuální stav ve Windows}
	\begin{itemize}
		\item{podpora pouze pro ANSI C 89}
		\item{pouze Windows API}
	\end{itemize}
\end{frame}

\begin{frame}
	\frametitle{Unixový svět}
	\begin{itemize}
		\item{norma POSIX}
		\begin{itemize}
			\item{kterou musí každý certifikovaný UNIX splňovat}
			\item{jazyk C je součástí POSIXu}
		\end{itemize}
		\item{historie}
		\begin{itemize}
			\item{POSIX.1 (1988)}
			\item{POSIX:2008 (aktuální norma)}
		\end{itemize}
	\end{itemize}
\end{frame}

\subsection{Further reading...}
\begin{frame}
	\frametitle{Doporučené čtení}
		\begin{center}
	\includegraphics[height=60mm]{img/unix-kniha.jpg}
	\end{center}
\end{frame}

\section{Práce s OS v jazyce C}

\begin{frame}
	\frametitle{Práce s OS v jazyce C}
	\tableofcontents[currentsection]
\end{frame}

\subsection{Podpora v jazyce}

\begin{frame}
	\frametitle{Manipulace se soubory}
	\begin{itemize}
		\item{bloková a textová práce se soubory}
		\item{manipulace se samotnými soubory \texttt{stdio.h}}
		\begin{itemize}
			\item{\texttt{FILE* tmpfile(void);}}
			\item{\texttt{int rename(const char*, const char*);}}
			\item{\texttt{int remove(const char*);}}
		\end{itemize}
	\end{itemize}
\end{frame}

\begin{frame}
	\frametitle{Práce s časem}
	\begin{itemize}
		\item{unixový čas \texttt{time\_t}}
		\begin{itemize}
			\item{počet vteřin od 1.1. 1970}
			\item{\texttt{time\_t time(time\_t *tloc);}}
		\end{itemize}
		\item{struktura pro ukládání času \texttt{struct tm}}
		\begin{itemize}
			\item{\texttt{time\_t} můžeme převést na tuto strukturu}
			\item{\texttt{struct tm *gmtime(const time\_t *clock);}}
			\item{\texttt{struct tm *localtime(const time\_t *clock);}}
		\end{itemize}
		\item{textové reprezentace}
		\begin{itemize}
			\item{\texttt{char *ctime(const time\_t *clock);}}
			\item{\texttt{char *asctime(const struct tm *tm);}}
		\end{itemize}
	\end{itemize}
\end{frame}

\begin{frame}
	\frametitle{Práce se signály}
	\begin{itemize}
		\item{\texttt{int raise(signal sig);}}
		\item{\texttt{void (*signal(int sig, void (*func)(int)))(int);}\footnote{demo \texttt{signals.c}}}
	\end{itemize}
\end{frame}

\begin{frame}
	\frametitle{Spouštění externích programů}
	\begin{itemize}
		\item{základní podpora}
		\item{synchronní, blokující}
		\begin{itemize}
			\item{\texttt{int system(const char* cmd);}\footnote{demo \texttt{ls.c}}}
		\end{itemize}
	\end{itemize}
\end{frame}

\begin{frame}
	\frametitle{Práce s locales}
	\begin{itemize}
		\item{možnost práce s locales}
		\item{samotné locales jsou platformně závislé}
		\item{\texttt{char *setlocale(int cat, const char* locale);}\footnote{locale.c}}
		\item{\texttt{struct lconv *localeconv(void);}}
	\end{itemize}
\end{frame}

\subsection{POSIX}

\begin{frame}
	\frametitle{POSIX}
	\begin{itemize}
		\item{rozšíření standardní knihovny C}
		\begin{itemize}
			\item{garance thread safe}
			\item{reentrant verze funkcí}
			\item{nové funkce}
		\end{itemize}
		\item{systémové operace}
		\begin{itemize}
			\item{API varianty všech cmdline příkazů}
			\item{práce s vlákny}
			\item{práce s procesy}
			\item{komunikace (lokální i síť)}
			\item{a další....}
		\end{itemize}
	\end{itemize}
\end{frame}

\begin{frame}
	\frametitle{Rozšíření standardní knihovny}
	\begin{itemize}
		\item{množství nových funkcí}
		\item{\texttt{ssize\_t getline(char **line, size\_t *n, FILE *);}\footnote{demo \texttt{posix.c}}}
		\item{\texttt{char *strndup(const char *s, size\_t n);}}
	\end{itemize}
\end{frame}

\begin{frame}
	\frametitle{Systémové operace}
	\begin{itemize}
		\item{API varianty unixových příkazů}
		\begin{itemize}
			\item{\texttt{int nice(int incr);}\footnote{demo \texttt{nice.c}}}
			\item{\texttt{int kill(pid\_t pid, int sig);}}
			\item{\texttt{int chmod(const char *path, mode\_t mode);}}
		\end{itemize}
		\item{práce s filesystémem}
		\item{\texttt{dirent.h}\footnote{demo \texttt{dirent.c}}}
		\item{\texttt{sys/stat.h}\footnote{demo \texttt{stat.c}}}
	\end{itemize}
\end{frame}

\begin{frame}
	\frametitle{Spouštění programů}
	\begin{itemize}
		\item{asynchronní, možnost komunikace}
		\begin{itemize}
			\item{\texttt{FILE* popen(const char* cmd, const char* mode);}\footnote{demo \texttt{calc.c}}\footnote{demo \texttt{size.c}}}
			\item{\texttt{int pclose(FILE*);}}
		\end{itemize}
	\end{itemize}
\end{frame}

\begin{frame}
	\frametitle{Rodina \texttt{fork}/\texttt{exec}}
	\begin{itemize}
		\item{rodina funkcí pro vytváření a správu procesů\footnote{demo \texttt{watcher.c} \texttt{pocet.c} \texttt{fork.c}}}
		\item{\texttt{pid\_t fork(void);}}
		\item{\texttt{pid\_t wait(int *status);}}
		\item{\texttt{pid\_t waitpid(pid\_t pid, int *stat, int opt);}}
		\item{\texttt{int dup(int fildes);}}
		\item{\texttt{int dup2(int fildes, int fildes2);}}
		\item{...}
	\end{itemize}
\end{frame}

\begin{frame}
	\frametitle{Doporučené čtení}
	\begin{itemize}
		\item{manuálové stránky}
		\begin{itemize}
			\item{\texttt{man funkce}}
			\item{\texttt{man soubor.h}}
		\end{itemize}
		\item{Wikipedie}
	\end{itemize}
\end{frame}

\section{Kontaktní informace}
	\subsection{Pracovní}

\begin{frame}[label=kontakt-simontoth]
	\frametitle{Mgr. Šimon Tóth}
	\begin{itemize}
		\item{Gotex (Šumavská 15) - kanceláře Cesnet (B310)}
	\end{itemize}

	\begin{columns}
	\column{0.5\textwidth}
		Fakulta Informatiky (MU)
		\begin{itemize}
			\item{toth@fi.muni.cz}
			\item{tel. 549 49 6446}
		\end{itemize}

	\column{0.5\textwidth}
		Cesnet z.s.p.o.
		\begin{itemize}
			\item{simon@cesnet.cz}
			\item{kancelář C122}
			\item{tel. 234 680 235}
		\end{itemize}

	\end{columns}
\end{frame}

	\subsection{Osobní}

\begin{frame}
	\frametitle{Mgr. Šimon Tóth}
		Osobní
		\begin{itemize}
			\item{kontakt@simontoth.cz}
			\item{tel. 776 565 424}
		\end{itemize}
\end{frame}



\end{document}
