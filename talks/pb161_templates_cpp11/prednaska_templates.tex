\documentclass[pdftex,aspectratio=169]{beamer}

% Setup beamer look theme
\mode<presentation>
{
	\useinnertheme{rectangles}
	\useoutertheme{infolines}
	\usecolortheme{crane}
	\setbeamertemplate{navigation symbols}{}
}

% Use nicer font
\usepackage{palatino}
\usepackage{palatino,graphics,array}

% Setup UTF-8 encoding
\usepackage[utf8]{inputenc}
\usepackage[T1]{fontenc}

% Setup czech language
\usepackage[czech]{babel}

% Include usefull packages
\usepackage{graphics}

% Include hyperlinks support
\usepackage{hyperref}
\hypersetup{colorlinks=true,linkcolor=black,urlcolor=black,unicode=true}

% Add support for simple URLs
\usepackage{url}

% Commands for CC license
\newcommand{\CcLongnameByNcSa}{\href{http://creativecommons.org/licenses/by-nc-sa/3.0/cz/}{Attribution-Noncommercial-ShareAlike}}
\newcommand{\CcImageBy}[1]{\includegraphics[scale=#1]{../commons/graphics/cc-by-white.pdf}}
\newcommand{\CcImageNc}[1]{\includegraphics[scale=#1]{../commons/graphics/cc-nc-white.pdf}}
\newcommand{\CcImageSa}[1]{\includegraphics[scale=#1]{../commons/graphics/cc-sa-white.pdf}}
\newcommand{\CcGroupByNcSa}[2]{\CcImageBy{#1}\hspace*{#2}\CcImageNc{#1}\hspace*{#2}\CcImageSa{#1}}




\title{Šablony a nový standard C++}
\subtitle{Přednášeno v rámci předmětu PB161}
\author[]{Mgr.~Šimon~Tóth}
\institute[FI@MU]{Fakulta informatiky @ Masarykova Univerzita}
\date{\today}

\newcommand{\CcNote}[1]{% longname
        Licencováno pod: \textit{Creative Commons #1 3.0 License}.%
}

\begin{document}

\begin{frame}
\titlepage
	\vfill
	\begin{center}
		\CcGroupByNcSa{0.33}{0.95ex}\\ {\tiny\CcNote{\CcLongnameByNcSa}}
		\vspace*{2ex}
	\end{center}
\end{frame}

\section{Úvod}
\subsection{Obsah přednášky}

\begin{frame} \frametitle{Obsah přednášky}
	\begin{itemize}
		\item{pokročilá práce se šablonami} \pause
		\item{upravené pro C++11} \pause
		\item{krátký přehled nových vlastností jazyka}
	\end{itemize}
\end{frame}

\begin{frame} \frametitle{Šablony}
	\begin{itemize}
		\item{zopakování šablon}
		\begin{itemize}
			\item{šablony funkcí}
			\item{šablony typů}
			\item{částečné specializace}
		\end{itemize}
		\item{pokročilá práce se šablonami}
		\begin{itemize}
			\item{metaprogramování}
			\item{šablony v kontextu C++11}
			\item{šablony a sémantika}
		\end{itemize}
	\end{itemize}
\end{frame}

\begin{frame} \frametitle{C++11}
	\begin{itemize}
		\item{\textbf{TODO}}
	\end{itemize}
\end{frame}

\subsection{Opakování šablon}

\begin{frame} \frametitle{Stupně generičnosti}
	\begin{itemize}
		\item{typově nezávislý kód}
		\begin{itemize}
			\item{reprezentace algoritmu}
			\begin{itemize}
				\item{konstanta \textit{(výpočet $\pi$)}} \pause
				\item{tabulka hodnot \textit{(goniometrické funkce)}} \pause
				\item{funkce} \pause
				\item{šablona funkce} \pause
			\end{itemize}
			\item{reprezentace datové struktury}
			\begin{itemize}
				\item{proměnná} \pause
				\item{datový typ} \pause
				\item{šablona}
			\end{itemize}
		\end{itemize}
	\end{itemize}
\end{frame}

\begin{frame} \frametitle{Základní syntax}
	\begin{overlayarea}{\textwidth}{2cm}
		\only<1>{\texttt{\textbf{template} < \textbf{typename} paramName >\\ void func(paramName\& p) \{\}}}
		\only<2>{\texttt{\textbf{template} < \textbf{typename} paramName >\\ \textbf{class} Class \{\};}}
		\only<3>{\texttt{\textbf{template} < \textbf{int} size >\\ \textbf{class} Class \{\};}}
		\only<4>{\texttt{\textbf{template} < \textbf{int} p1, \textbf{typename} p2, \textbf{class} p3 >\\ \textbf{class} Class \{\};}}
		\only<5>{\texttt{\textbf{template} < \textbf{int} p1 = 3, \textbf{typename} p2 = double, \\
		\textbf{class} p3 = std::string >\\ \textbf{class} Class \{\};}}
	\end{overlayarea}
	\begin{overlayarea}{\textwidth}{2cm}
		\only<1>{hlavička šablony\\ následovaná definicí funkce}
		\only<2>{hlavička šablony\\ následovaná definicí funkce nebo šablonového typu}
		\only<3>{parametrem může být i ordinální hodnota}
		\only<4>{a samozřejmě, parametrů může být více}
		\only<5>{a můžou mít defaultní hodnoty}
	\end{overlayarea}
\end{frame}

\begin{frame} \frametitle{Šablony funkcí}
	\begin{itemize}
		\item{jednodušší varianta šablon}
		\item{umožňuje psát bezpečné generické funkce}
		\item{pokud je to možné, kompilátor sám zjistí parametry šablony}
	\end{itemize}
	\textbf{Bezpečnost:}
	\begin{itemize}
		\item{parametry se vyhodnocují (vs. makra)\footnote{\texttt{macro.cpp}}}
		\item{zachovává se silná typovost (vs. \texttt{void*})\footnote{\texttt{swap.cpp}}}
		\item{instance šablony se vytvoří pouze tehdy, je-li to nutné}
	\end{itemize}
	\textbf{Extenzibilita:}
	\begin{itemize}
		\item{pomocí úplných specializací můžeme specifikovat výjimky\footnote{\texttt{swap.cpp}}}
	\end{itemize}
\end{frame}




\if{0}




\subsection{Opakování šablon}


\subsection{Základní syntax}


	\subsection{Šablony funkcí}


	\begin{frame}
		\frametitle{Pořadí vyhodnocování}
		Co se stane, když kompilátor narazí na volání funkce?\footnote{\texttt{004\_template\_function\_order.cpp}}
		\begin{enumerate}
			\item{Pokud existuje klasická funkce, která splňuje prototyp, použije se.}\\
		\textbf{else}
			\item{Pokud existuje vyhovující úplná specializace šablony, použije se.}\\
		\textbf{else}
			\item{Kompilátor vytvoří instanci šablony pro daný prototyp.}
		\end{enumerate}
	\end{frame}

	\begin{frame}
		\frametitle{Šablony metod}
		\begin{itemize}
			\item{metoda je funkce s jedním skrytým parametrem}
			\item{aplikují se stejná pravidla jako u funkcí\footnote{\texttt{005\_template\_method.cpp}}}
		\end{itemize}
	\end{frame}

	\subsection{Šablony tříd}

	\begin{frame}
		\frametitle{Šablony tříd}
		\begin{itemize}
			\item{umožňují konstrukci generických typů\footnote{\texttt{006\_template\_class.cpp}}}
			\item{typově nezávislý způsob ukládání dat + připojené algoritmy}
			\item{metody automaticky přejímají šablonu třídy}
			\item{podpora částečného vyhodnocování}
			\item{protože kompilátor nemá podle čeho určit parametry, musíme je explicitně uvést}
			\item{vzhledem k vyhodnocení v době kompilace použitelné pro metaprogramování}
			\item{STL}
		\end{itemize}
	\end{frame}

	\begin{frame}
		\frametitle{Částečné specializace}
		\begin{itemize}
			\item{velice podobný princip jako u úplných specializací\footnote{\texttt{007\_template\_class\_partial.cpp}}}
			\item{nyní ale fixujeme pouze některé parametry}
			\item{používá se hlavně k metaprogramování}
		\end{itemize}
	\end{frame}

	\begin{frame}
		\frametitle{Přátelé a šablony}
		\begin{itemize}
			\item{jednotlivé instance šablony jsou samostatné typy}
			\item{nemají přístup k privátním a chráněným položkám jiných instancí\footnote{\texttt{008\_template\_friend.cpp}}}
		\end{itemize}
	\end{frame}


	\begin{frame}
		\frametitle{Praktické využití částečné specializace}
		\textbf{\texttt{TypeSelector}}\\
		\begin{itemize}
			\item{někdy potřebujeme vybrat konkrétní typ na základě vlastností platformy}
			\item{můžeme použít klasické konstrukce \texttt{\#if \#else \#endif}}
			\item{šablony jsou ale elegantnější\footnote{\texttt{009\_template\_type\_selector.cpp}}}
		\end{itemize}
	\end{frame}

	\subsection{Další vlastnosti}

	\begin{frame}
		\frametitle{Defaultní hodnoty parametrů}
		\begin{itemize}
			\item{je možné specifikovat defaultní hodnoty parametrů šablony\footnote{\texttt{010\_template\_default.cpp}}}
			\item{fungují stejná pravidla jako u parametrů funkcí}
		\end{itemize}
	\end{frame}

	\begin{frame}
		\frametitle{Šablona parametrizovaná šablonou}
		\begin{itemize}
			\item{parametrem šablony může být i jiná šablona\footnote{\texttt{011\_template\_templated.cpp}}}
			\item{ta je pak dostupná v třídě stejně jako jiné parametry}
			\item{protože jde o šablonu, musíme ji instanciovat}
		\end{itemize}
	\end{frame}

	\section{Metaprogramování}
	\subsection{Úvod}

	\begin{frame}
		\frametitle{Metaprogramování}
		\begin{itemize}
			\item{psaní programů prováděných při kompilaci}
			\item{převážně založeno na šablonách}
			\item{možné využít i preprocesor}
			\item{otevírá nové možnosti návrhu softwaru}
			\item{velice podobné logickému programování}
		\end{itemize}
	\end{frame}

	\subsection{Výpočetní metaprogramy}

	\begin{frame}
		\frametitle{Výpočetní metaprogramy}
		\begin{itemize}
			\item{výsledkem výpočetního metaprogramu je jedna konstanta}
			\item{větší přenositelnost a flexibilita než preprocesor}
		\end{itemize}\pause
		\textbf{Klasický problém:}\\
		Bylo zjištěno, že při výpočtu algoritmu bude potřeba pole integerů odpovídající faktoriálu velikosti typu \texttt{int} na dané platformě.
	\end{frame}

	\begin{frame}
		\frametitle{Výpočet faktoriálu}
		\begin{itemize}
			\item{použijeme šablony a jednoduchý rekurzivní algoritmus}
			\item{ukončovací podmínku vyjádříme pomocí specializace\footnote{\texttt{012\_meta\_factorial.cpp}}}
		\end{itemize}
	\end{frame}

	\begin{frame}
		\frametitle{Výpočet celočíselné odmocniny}
		\begin{itemize}
			\item{použijeme algoritmus půlení intervalu}
			\item{ukončovací podmínku (interval je délky nula) vyjádříme opět pomocí specializace}
			\item{kvůli efektivitě použijeme \texttt{TypeSelector}\footnote{\texttt{013\_meta\_sqrt.cpp}}}
		\end{itemize}
	\end{frame}

	\subsection{Klasické metaprogramy}

	\begin{frame}
		\frametitle{Klasické metaprogramy}
		\begin{itemize}
			\item{statické náhrady za klasické konstrukce jazyka}
			\item{vše je vyhodnoceno v době kompilace}
			\item{tvorba efektivních programů}
		\end{itemize}
	\end{frame}

	\begin{frame}
		\frametitle{Statické konstrukce jazyka}
		\textbf{Statický if}
		\begin{itemize}
			\item{jednoduchá varianta\footnote{\texttt{014\_meta\_if.cpp}}}
			\item{pokročilá varianta\footnote{\texttt{015\_meta\_if\_enh.cpp}}}
		\end{itemize}
		\textbf{Statický switch}
		\begin{itemize}
			\item{rozšíření ifu\footnote{\texttt{016\_meta\_switch.cpp}}}
		\end{itemize}
		\textbf{Statický cyklus}
		\begin{itemize}
			\item{různé možnosti implementace\footnote{\texttt{017\_meta\_cycle.cpp}}}
		\end{itemize}
	\end{frame}

	\subsection{Efektivita šablon}

	\begin{frame}
		\frametitle{Šablony podporující efektivitu}
		\begin{itemize}
			\item{šablonové funkce, metody a třídy se vytvoří, pouze pokud se použijí}
			\item{díky informacím v době kompilace mohou být provedeny lepší optimalizace}
		\end{itemize}
	\end{frame}

	\begin{frame}
		\frametitle{Šablony a kompilace}
		\begin{itemize}
			\item{šablony zásadně prodlužují délku kompilace}
			\item{díky způsobu vyhodnocování je ale vícenásobné použití zadarmo}
		\end{itemize}
	\end{frame}

	\begin{frame}
		\frametitle{Šablony a velikost binárky}
		\begin{itemize}
			\item{vygenerované typy zabírají v binárce místo}
			\item{je to daň za výpočetní efektivitu}
			\item{\texttt{strip --strip-uneeded soubor} pomůže}
		\end{itemize}
	\end{frame}

	\section{Tipy a triky}
	\subsection{Rozšířené rozhraní}

	\begin{frame}
		\frametitle{Rozšířené rozhraní}
		\begin{itemize}
			\item{využití faktu, že se šablony instanciují pouze, pokud se použijí}
			\item{platí i pro metody v šablonované třídě}
			\item{některé metody můžou být typově specifické\footnote{\texttt{018\_tricks\_extended.cpp}}}
		\end{itemize}
	\end{frame}

	\subsection{Curiously recurring pattern}

	\begin{frame}
		\frametitle{Curiosly recurring pattern}
		\begin{itemize}
			\item{významný dopad na návrh software}
			\item{předek \textit{zná} svého potomka\footnote{\texttt{019\_tricks\_crp.cpp}}}
			\item{oddělení typově nezávislé funkčnosti\footnote{\texttt{020\_tricks\_crp\_use.cpp}}}
		\end{itemize}
	\end{frame}

	\subsection{Statický polymorfizmus}

	\begin{frame}
		\frametitle{Statický polymorfizmus}
		\begin{itemize}
			\item{podobná funkčnost jako virtuální funkce}
			\item{vyhodnocovaná v době překladu\footnote{\texttt{021\_tricks\_static\_polymorph.cpp}}}
		\end{itemize}
	\end{frame}

	\subsection{Template injection}

	\begin{frame}
		\frametitle{Template injection}
		\begin{itemize}
			\item{řešení multiplikace kódu u některých operátorů\footnote{\texttt{022\_tricks\_inject.cpp}}\footnote{\texttt{023\_tricks\_inject2.cpp}}}
		\end{itemize}
	\end{frame}

	\section{Návrh založený na zásadách}
	\subsection{Ideální svět}

	\begin{frame}
		\frametitle{Ideální svět}
		\begin{itemize}
			\item{maximální znovu-použitelnost kódu}
			\item{možné jednoduše kombinovat kusy kódu}
			\item{\textit{dočasný zaměstnanec + sekretářka = dočasná sekretářka}}
		\end{itemize}
	\end{frame}

	\begin{frame}
		\frametitle{Vícenásobná dědičnost}
		\begin{itemize}
			\item{něco podobného slibuje}
			\item{kombinace ale není přímočará}
			\item{nutná spousta lepícího kódu}
		\end{itemize}
	\end{frame}

	\subsection{Vícenásobná dědičnost}

	\begin{frame}
		\frametitle{Co je špatně?}
		\begin{itemize}
			\item{vícenásobná dědičnost je považována za špatný návrhový vzor}
			\item{spousta jazyků ji vůbec nepodporuje}
		\end{itemize}
	\end{frame}

	\begin{frame}
		\frametitle{Co je \textbf{skutečně} špatně?}
		\begin{itemize}
			\item{vícenásobná dědičnost si vynucuje některé nehezké vlastnosti}
			\item{může mít nepříjemné vedlejší efekty}
		\end{itemize}
	\end{frame}

	\begin{frame}
		\frametitle{Konflikty jmen a duplikace atributů}
		\begin{itemize}
			\item{kolize jmen\footnote{\texttt{024\_contract\_kolize.cpp}}}
			\item{problém diamantu\footnote{\texttt{025\_contract\_diamant.cpp}}}
			\item{řešením je obvykle virtuální dědičnost\footnote{\texttt{026\_contract\_virtual.cpp}}}
		\end{itemize}
	\end{frame}

	\begin{frame}
		\frametitle{Virtuální dědičnost je problematická}
		\begin{itemize}
			\item{virtuální dědičnost je pomalá}
			\item{mění pořadí volání konstruktorů\footnote{\texttt{027\_contract\_constructors.cpp}}}
			\item{\textbf{musíme ji použít dřív, než ji potřebujeme}}
		\end{itemize}
	\end{frame}

	\begin{frame}
		\frametitle{Co s tím?}
		\begin{itemize}
			\item{problémy způsobuje \textit{hluboká} dědičnost}
			\item{kdybychom se jí dokázali vyhnout, vyřešíme skoro vše}
		\end{itemize}\pause
		\textbf{Kde je problém?}
		\begin{itemize}
			\item{inteligenci dodávají vrchní třídy}
			\item{ty potřebují, aby datové složky byly pod nimi}
			\item{při ploché dědičnosti budou vedle sebe}
			\item{šablony nám pomohou}
		\end{itemize}
	\end{frame}

	\begin{frame}
		\frametitle{Návrh založený na zásadách}
		\begin{itemize}
			\item{dědičnost je sama o sobě příliš slabá}
			\item{zkombinujeme se šablonami}
			\item{\textit{curiously recurring template pattern}}
		\end{itemize}
	\end{frame}

	\subsection{Zásady}

	\begin{frame}
		\frametitle{Zásady}
		\begin{itemize}
			\item{implementovanou funkčnost rozdělíme na ortogonální zásady}
			\item{\textit{typ kontroly, způsob alokace, formát načítaných dat...}}
			\item{pro každou zásadu implementujeme všechny potřebné varianty\footnote{\texttt{028\_contract\_zasada.cpp}}}
		\end{itemize}
	\end{frame}

	\subsection{Třída zásad}

	\begin{frame}
		\frametitle{Třída zásad}
		\begin{itemize}
			\item{kostra pro zásady}
			\item{obvykle velice jednoduchý spojovací kód}
			\item{může obsahovat rozšířené metody\footnote{029\_contract\_trida\_zasad.cpp}}
		\end{itemize}
	\end{frame}

	\subsection{Knihovna Loki}

	\begin{frame}
		\frametitle{Knihovna loki}
		\begin{itemize}
			\item{knihovna využívající návrh založený na zásadách}
			\item{\url{http://loki-lib.sourceforge.net/}}
		\end{itemize}
	\end{frame}

	\section{Závěr}

	\begin{frame}
		\frametitle{Děkuji za pozornost}
		\begin{flushleft}
			\begin{huge}The truth is out there...\end{huge}
		\end{flushleft}
		\begin{flushright}
			\begin{huge}...seek it.\end{huge}
		\end{flushright}
	\end{frame}


% constexpr
% type inference - auto, decltype(X)
% variadric templates
% brackets
% template aliases
% static assertions
% type traits


% initializer lists
% uniform initialization
% retezove vyvolavani konstruktoru
% final & override
% explicit conversion operators


% r-value references

% range loops
% lambda functions & algorithms

% nullptr

% strongly typed enumerations

% string literals, & custom literals

% smart pointers



\fi
\end{document}
