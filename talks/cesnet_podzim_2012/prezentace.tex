\documentclass[pdftex,aspectratio=169]{beamer}

% Setup beamer look theme
\mode<presentation>
{
	\useinnertheme{rectangles}
	\useoutertheme{infolines}
	\usecolortheme{beaver}
	\setbeamertemplate{navigation symbols}{}
}

% Use nicer font
\usepackage{palatino,graphics,array}

% Setup UTF-8 encoding
\usepackage[utf8]{inputenc}
\usepackage[T1]{fontenc}

% Setup czech language
\usepackage[czech]{babel}

% Include usefull packages
\usepackage{graphics}

% Include hyperlinks support
\usepackage{hyperref}
\hypersetup{colorlinks=true,linkcolor=black,urlcolor=black,unicode=true}

% Add support for simple URLs
\usepackage{url}

\title{Torque}
\subtitle{co se tento rok stalo a stane}
\author[]{Mgr.~Šimon~Tóth}
\date{\today}

\begin{document}

\begin{frame}
	\titlepage
\end{frame}

\section{Přehled}
\subsection{Tento rok}

\begin{frame}
	\frametitle{Co se stalo tento rok}
	\pause
	\begin{itemize}
		\item{převážně ladění}
		\begin{itemize}
			\item{odstraňování všemožných chyb}
			\item{ladění sémantiky}
			\item{spíše drobné vlastnosti}
		\end{itemize}
	\end{itemize}
\end{frame}

\subsection{Do konce roku}

\begin{frame}
	\frametitle{Na čem se aktivně pracuje}
	\pause
	\begin{itemize}
		\item{kompletní refactoring kódu plánovače}
		\item{eliminace těžce spravovatelného kódu}
		\item{půda pro nové vlastnosti}
	\end{itemize}
\end{frame}

\section{Tento rok v detailech}
\subsection{Přehled}

\begin{frame}
	\frametitle{Přehled}
	\begin{itemize}
		\item{nové vlastnosti, převážně zaměřené na user support}
		\item{větší vlastnosti čekají rozpracované ve frontě}
	\end{itemize}
\end{frame}

\subsection{Kontrola limitů zdrojů}

\begin{frame}
	\frametitle{Kontrola limitů zdrojů}
	\begin{itemize}
		\item{kontrola CPU a paměti}
		\item{silně konfigurovatelné}
		\item{podpora pro exkluzivní a (osamocené) joby}
		\item{v provozu už delší dobu}
		\item{ve frontě ještě změna kontroly na časové úseky}
	\end{itemize}
\end{frame}

\begin{frame}
	\frametitle{Exportace limitů do environmentu}
	\begin{itemize}
		\item{řešení problému heterogenních strojů}
		\item{uživatelé můžou použít k nastavení limitů SW}
	\end{itemize}
\end{frame}

\subsection{Archivace jobů}

\begin{frame}
	\frametitle{Archivace jobů}
	\begin{itemize}
		\item{joby se mažou ze serveru 24 hodin po skončení}
		\item{to je málo pro řešení problémů uživatelů}
		\begin{itemize}
			\item{kam mi zmizel job?}
			\item{proč mi selhal job?}
			\item{...}
		\end{itemize}
		\item{všechny joby se nyní archivují}
		\item{prozatím pouze dostupné na serveru přes ssh}
	\end{itemize}
\end{frame}

\subsection{Limitace front podle CPU}

\begin{frame}
	\frametitle{Limitace front podle CPU}
	\begin{itemize}
		\item{limitace fronty podle počtu využitých CPU}
		\item{limitace uživatele ve frontě pole počtu využitých CPU}
		\item{limitace skupiny ve frontě podle počtu využitých CPU}
	\end{itemize}
\end{frame}

\subsection{Drobná vylepšení}

\begin{frame}
	\frametitle{Licence}
	\begin{itemize}
		\item{výrazně se eliminoval problém s plánováním licencí}
		\item{pořád máme problém u Matlabu}
		\begin{itemize}
			\item{velký tlak externích uživatelů}
		\end{itemize}
	\end{itemize}
\end{frame}

\begin{frame}
	\frametitle{Změna fairshare}
	\begin{itemize}
		\item{fairshare se počítá podle statické alokace}
		\begin{itemize}
			\item{\texttt{walltime * ncpus}}
		\end{itemize}
	\end{itemize}
\end{frame}

\begin{frame}
	\frametitle{Environment injection}
	\begin{itemize}
		\item{provozní vlastnost}
		\item{možnost vecpat jobu dynamicky generovaný environment}
		\item{konfigurovatelné per-stroj}
	\end{itemize}
\end{frame}

\begin{frame}
	\frametitle{LB}
	\begin{itemize}
		\item{ze strany Torque jde o vyřešený problém}
		\begin{itemize}
			\item{modulo pár záhadných pádů}
		\end{itemize}
	\end{itemize}
\end{frame}

\section{Za rohem}
\subsection{Distribuovaný Torque}

\begin{frame}
	\frametitle{Distribuovaný Torque}
	\begin{itemize}
		\item{(snad) do konce měsíce}
		\item{pokud plánovač najde místo pro job na jiném serveru, spustí ho tam}
	\end{itemize}
\end{frame}

\begin{frame}
	\frametitle{Aktuální chování}
	\begin{itemize}
		\item{atomické spuštění}
		\item{originální job se přepne do speciálního stavu}
		\item{poznačí se cílový server}
		\item{podpora v qstatu}
		\begin{itemize}
			\item{pouze pro \texttt{qstat (-f) jobid}}
		\end{itemize}
	\end{itemize}
\end{frame}

\begin{frame}
	\frametitle{Nedořešené problémy}
	\begin{itemize}
		\item{fairshare}
		\begin{itemize}
			\item{můžeme aproximovat}
		\end{itemize}
		\item{strádání}
		\begin{itemize}
			\item{můžeme simulovat}
		\end{itemize}
	\end{itemize}
\end{frame}

\subsection{Scratch}

\begin{frame}
	\frametitle{Scratch}
	\begin{itemize}
		\item{heterogenní scratche}
		\begin{itemize}
			\item{obyčejný lokální}
			\item{SSD lokální}
			\item{síťový}
		\end{itemize}
	\end{itemize}
\end{frame}

\begin{frame}
	\frametitle{Logika plánovače}
	\begin{itemize}
		\item{pro uživatele se vybere scratch který má dost místa}
		\item{pokud si uživatel explicitně řekne o konkrétní scratch, job se spustí jenom když bude místo}
	\end{itemize}
\end{frame}

\subsection{Plánování podle výkonu strojů}

\begin{frame}
	\frametitle{Plánování podle výkonu počítačů}
	\begin{itemize}
		\item{nová vlastnost na strojích}
		\item{uživatel může zadat minimální výkon stroje}
	\end{itemize}
\end{frame}

\subsection{On-demand clustery}

\begin{frame}
	\frametitle{On-demand clustery}
	\begin{itemize}
		\item{automatické rebootování strojů na image který požadují joby}
		\item{automaticky se vytvoří nový cluster job}
		\item{postaví se speciální typ clusteru}
		\item{na clusteru se pak může pustit jak job pro který se postavil, tak jiný}
	\end{itemize}
\end{frame}

\section{Dlouhodobější výhled}
\subsection{Informace uživatelům}

\begin{frame}
	\frametitle{Informace uživatelům}
	\begin{itemize}
		\item{na co job čeká}
		\item{kdy se asi spustí}
	\end{itemize}
\end{frame}

\begin{frame}
	\frametitle{Detekce nespustitelných jobů}
	\begin{itemize}
		\item{plánovací algoritmus reportuje tři hodnoty}
		\begin{itemize}
			\item{může se spustit}
			\item{teď se nemůže spustit}
			\item{nikdy se nemůže spustit}
		\end{itemize}
		\item{otevřený problém kategorizace}
	\end{itemize}
\end{frame}

\begin{frame}
	\frametitle{Noví klienti}
	\begin{itemize}
		\item{přepis klientů}
		\begin{itemize}
			\item{\texttt{qstat}}
			\item{\texttt{qmgr}}
			\item{\texttt{pbsnodes}}
		\end{itemize}
		\item{informace z logiky plánovače}
		\item{detailnější dotazy}
		\begin{itemize}
			\item{volné stroje}
			\item{stroje dostupné z fronty X}
			\item{stroje odpovídající nodespec}
		\end{itemize}
		\item{strojově parsovatelný výstup}
	\end{itemize}
\end{frame}

\begin{frame}
	\frametitle{To určitě není vše}
	\begin{itemize}
		\item{pokračování v bugfixování}
		\item{další, operativně řešené, drobné vlastnosti}
	\end{itemize}
\end{frame}

\end{document}
