\documentclass[pdftex,aspectratio=169]{beamer}

% Setup beamer look theme
\mode<presentation>
{
	\useinnertheme{rectangles}
	\useoutertheme{infolines}
	\usecolortheme{beaver}
	\setbeamertemplate{navigation symbols}{}
}

% Use nicer font
\usepackage{palatino}
\usepackage{palatino,graphics,array}

% Setup UTF-8 encoding
\usepackage[utf8]{inputenc}
\usepackage[T1]{fontenc}

% Setup czech language
\usepackage[czech]{babel}

% Include usefull packages
\usepackage{graphics}

% Include hyperlinks support
\usepackage{hyperref}
\hypersetup{colorlinks=true,linkcolor=black,urlcolor=black,unicode=true}

% Add support for simple URLs
\usepackage{url}

\title{Torque}
\subtitle{co se tento rok stalo a stane}
\author[]{Mgr.~Šimon~Tóth}
\date{\today}

\begin{document}

\begin{frame}
	\titlepage
\end{frame}

\section{Přehled}
\subsection{Tento rok}

\begin{frame}
	\frametitle{Co se stalo tento rok}
	\begin{itemize}
		\item{převážně ladění}
		\begin{itemize}
			\item{odstraňování všemožných chyb}
			\item{ladění sémantiky}
		\end{itemize}
	\end{itemize}
\end{frame}

\subsection{Do konce roku}

\begin{frame}
	\frametitle{Kompletní přepis plánovače}
	\begin{itemize}
		\item{eliminace těžce spravovatelného kódu}
		\item{příprava pro nové vlastnosti}
	\end{itemize}
\end{frame}

\section{Tento rok v detailech}
\subsection{Kontrola limitů zdrojů}

% limity zdroju se kontroluji na uzlech
% promenne pro limity zdroju v environmentu

% fairshare se nyni pocita jako walltime * pocet cpu

% environment injection into job

% moznost limitovat fronty podle poctu obsazenych CPU

% job archive

% vylepsene planovani licenci

% LB

\begin{frame}
	\frametitle{Kontrola limitů zdrojů}
\end{frame}

\subsection{Archivace jobů}

\begin{frame}
	\frametitle{Archivace jobů}
	\begin{itemize}
		\item{joby se mažou ze serveru 24 hodin po skončení}
		\item{tohle je málo pro řešení problémů uživatelů}
		\begin{itemize}
			\item{kam mi zmizel job?}
			\item{proč mi selhal job?}
			\item{...}
		\end{itemize}
		\item{všechny joby se nyní archivují}
		\item{prozatím pouze dostupné na serveru přes ssh}
	\end{itemize}
\end{frame}

\section{Za rohem}
\subsection{Distribuovaný Torque}

\begin{frame}
	\frametitle{Distribuovaný Torque}
	\begin{itemize}
		\item{(snad) do konce měsíce}
		\item{pokud plánovač najde místo pro job na jiném serveru, spustí ho tam}
	\end{itemize}
\end{frame}

\begin{frame}
	\frametitle{Aktální chování}
	\begin{itemize}
		\item{pokud se povede job spustit}
		\item{originální job se přepne do speciálního stavu}
		\item{poznačí se cílový server}
		\item{podpora v qstatu}
		\begin{itemize}
			\item{pouze pro \texttt{qstat (-f) jobid}}
		\end{itemize}
	\end{itemize}
\end{frame}

\begin{frame}
	\frametitle{Nedořešené problémy}
	\begin{itemize}
		\item{fairshare}
		\begin{itemize}
			\item{můžeme aproximovat}
		\end{itemize}
		\item{strádání}
		\begin{itemize}
			\item{můžeme simulovat}
		\end{itemize}
	\end{itemize}
\end{frame}

\subsection{Scratch}

\begin{frame}
	\frametitle{Scratch}
	\begin{itemize}
		\item{heterogenní scratche}
		\begin{itemize}
			\item{obyčejný lokální}
			\item{SSD lokální}
			\item{síťový}
		\end{itemize}
	\end{itemize}
\end{frame}

\begin{frame}
	\frametitle{Logika plánovače}
	\begin{itemize}
		\item{pro uživatele se vybere scratch který má dost místa}
		\item{pokud si uživatel explicitně řekne o konkrétní scratch, job se spustí jenom když bude místo}
	\end{itemize}
\end{frame}

\subsection{Plánování podle výkonu strojů}

\begin{frame}
	\frametitle{Plánování podle výkonu počítačů}
	\begin{itemize}
		\item{nová vlastnost na strojích}
		\item{uživatel může zadat minimální výkon stroje}
	\end{itemize}
\end{frame}

\end{document}
