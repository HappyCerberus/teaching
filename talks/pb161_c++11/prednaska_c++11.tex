\documentclass[pdftex,aspectratio=169]{beamer}

% Setup beamer look theme
\mode<presentation>
{
	\useinnertheme{rectangles}
	\useoutertheme{infolines}
	\usecolortheme{crane}
	\setbeamertemplate{navigation symbols}{}
}

% Use nicer font
\usepackage{palatino}
\usepackage{palatino,graphics,array}

% Setup UTF-8 encoding
\usepackage[utf8]{inputenc}
\usepackage[T1]{fontenc}

% Setup czech language
\usepackage[czech]{babel}

% Include usefull packages
\usepackage{graphics}

% Include hyperlinks support
\usepackage{hyperref}
\hypersetup{colorlinks=true,linkcolor=black,urlcolor=black,unicode=true}

% Add support for simple URLs
\usepackage{url}

% Commands for CC license
\newcommand{\CcLongnameByNcSa}{\href{http://creativecommons.org/licenses/by-nc-sa/3.0/cz/}{Attribution-Noncommercial-ShareAlike}}
\newcommand{\CcImageBy}[1]{\includegraphics[scale=#1]{../commons/graphics/cc-by-white.pdf}}
\newcommand{\CcImageNc}[1]{\includegraphics[scale=#1]{../commons/graphics/cc-nc-white.pdf}}
\newcommand{\CcImageSa}[1]{\includegraphics[scale=#1]{../commons/graphics/cc-sa-white.pdf}}
\newcommand{\CcGroupByNcSa}[2]{\CcImageBy{#1}\hspace*{#2}\CcImageNc{#1}\hspace*{#2}\CcImageSa{#1}}




\title{Nový standard C++11}
\subtitle{Přednášeno v rámci předmětu PB161}
\author[]{Mgr.~Šimon~Tóth}
\institute[FI@MU]{Fakulta informatiky @ Masarykova Univerzita}
\date{\today}

\newcommand{\CcNote}[1]{% longname
        Licencováno pod: \textit{Creative Commons #1 3.0 License}.%
}

\begin{document}

\begin{frame}
\titlepage
	\vfill
	\begin{center}
		\CcGroupByNcSa{0.33}{0.95ex}\\ {\tiny\CcNote{\CcLongnameByNcSa}}
		\vspace*{2ex}
	\end{center}
\end{frame}

\section{Úvod}
\subsection{Obsah přednášky}

\begin{frame} \frametitle{Obsah přednášky}
	\begin{itemize}
		\item{aktuální stav standardu} \pause
		\item{novinky v syntaxi} \pause
		\item{novinky v knihovnách}
	\end{itemize}
\end{frame}

\begin{frame} \frametitle{Doporučené materiály}
	\begin{itemize}
		\item{\url{http://en.wikipedia.org/wiki/C++11}}
		\item{\url{http://cppreference.com}}
	\end{itemize}
\end{frame}

\subsection{Aktuální stav standardu}

\begin{frame} \frametitle{Aktuální stav standardu}
	\begin{itemize}
		\item{GCC}
		\begin{itemize}
			\item skoro kompletní implementace
			\item \url{http://gcc.gnu.org/gcc-4.8/cxx0x\_status.html}
		\end{itemize}

		\item{Clang}
		\begin{itemize}
			\item skoro kompletní implementace
			\item \url{http://clang.llvm.org/cxx\_status.html}
		\end{itemize}

		\item Visual Studio
		\begin{itemize}
			\item plná podpora knihovny
			\item nedostatky na úrovni základní syntaxe
		\end{itemize}
	\end{itemize}
\end{frame}

\section{Novinky v syntaxi}

\subsection{Nedostatky starého C++}

\begin{frame} \frametitle{\texttt{nullptr}}
	\begin{itemize}
		\item \texttt{NULL} je definován jako \texttt{0U}
		\item problém při přetěžování funkcí/metod
		\item C++11 přináší nové klíčové slovo \texttt{nullptr}\footnote{demo 1}
	\end{itemize}
\end{frame}

\begin{frame} \frametitle{Problémy s inicializací}
	\begin{itemize}
		\item inicializační syntax není uniformní
		\item některé způsoby inicializace nefungují v určitých kontextech
		\item některé způsoby inicializace nefungují s třídami
		\item C++11 přináší uniformní inicializaci\footnote{demo 2}
	\end{itemize}
\end{frame}

\begin{frame} \frametitle{Type inference}
	\begin{itemize}
		\item popsat typ proměnné může být občas problém
		\begin{itemize}
			\item typ je příliš dlouhý
			\item typ je těžké zjistit
		\end{itemize}
		\item C++11 pro tyto případy přináší inferenci typu\footnote{demo 3}
	\end{itemize}
\end{frame}

\subsection{Nové vlastnosti v C++11}

\begin{frame} \frametitle{Alternativní for-cyklus}
	\texttt{for (auto x : v) \{ \}\footnote{demo 4}}
\end{frame}

\begin{frame} \frametitle{Lambdy}
	\begin{itemize}
		\item používání STL algoritmů je v starém C++ nepohodlné
		\item ideální by byla možnost psát inline funkce na místě použití
		\item C++ přináší lambdy\footnote{demo 5}\footnote{demo 6}
	\end{itemize}
\end{frame}


\section{Novinky v knihovnách}

\subsection{Random}

\begin{frame} \frametitle{Random}
	\begin{itemize}
		\item Máme N karet, kolik různých sekvencí těchto karet jsme schopni vygenerovat pomocí funkce \texttt{rand()}?
		\begin{enumerate}
			\item $N!$
			\item $N^N$
			\item $1$
			\item $N^2$
			\item $N + N-1 + N-2 ...$
			\item ani jedna možnost není správně
		\end{enumerate}
	\end{itemize}
\end{frame}

\begin{frame} \frametitle{Random}
	\begin{itemize}
		\item funkce \texttt{rand()} je limitována počtem unikátních seed hodnot
		\item \texttt{void srand(unsigned seed);}
		\item $ min(N!,2^{sizeof(unsigned)*CHAR\_BIT}) $
	\end{itemize}
\end{frame}

\begin{frame} \frametitle{Random v C++11}
	\begin{itemize}
		\item podpora pro různé generátory
		\item podpora pro různé distribuce\footnote{demo 7}
	\end{itemize}
\end{frame}

\subsection{Datové struktury}

\begin{frame} \frametitle{\texttt{array}}
	\begin{itemize}
		\item statické pole
		\item STL rozhraní\footnote{demo 8}
	\end{itemize}
\end{frame}

\begin{frame} \frametitle{\texttt{forward\_list}}
	\begin{itemize}
		\item jednosměrný spojovaný seznam
	\end{itemize}
\end{frame}

\begin{frame} \frametitle{\texttt{unordered\_set}, \texttt{unordered\_map}}
	\begin{itemize}
		\item hash verze \texttt{set} a \texttt{map}
	\end{itemize}
\end{frame}

\begin{frame} \frametitle{\texttt{tuple}}
	\begin{itemize}
		\item nehomogenní datový typ
		\item rozšíření \texttt{pair} pro neomezený počet položek\footnote{demo 9}
	\end{itemize}
\end{frame}

\subsection{Algoritmy}

\begin{frame} \frametitle{Algoritmy}
	\begin{itemize}
		\item \texttt{all\_of}, \texttt{any\_of}, \texttt{none\_of}
		\item \texttt{shuffle}
		\item \texttt{copy\_if}
		\item \texttt{is\_sorted}, \texttt{is\_permutation}
	\end{itemize}
\end{frame}

\begin{frame} \frametitle{Bind}
	\begin{itemize}
		\item C++98 \texttt{bind1st}, \texttt{bind2nd}
		\item C++11 \texttt{bind} + placeholders\footnote{demo 10}\footnote{demo 11}
	\end{itemize}
\end{frame}

\subsection{Práce s časem}

\begin{frame} \frametitle{Práce s časem}
	\begin{itemize}
		\item podpora pro systémové, monotónní i high-res hodiny
		\item podpora pro časové úseky\footnote{demo 12}
	\end{itemize}
\end{frame}

\subsection{Chytré ukazatele}

\begin{frame} \frametitle{Chytré ukazatele}
	\begin{itemize}
		\item \texttt{unique\_ptr} - unikátní ukazatel, vlastní přidělenou paměť
		\item \texttt{weak\_ptr} - ukazatel bez držení vlastnictví\footnote{demo 13}
		\item \texttt{shared\_ptr} - sdílené vlastnictví paměti\footnote{demo 14}
	\end{itemize}
\end{frame}

\subsection{Asynchronní operace}

\begin{frame} \frametitle{Asynchronní operace}
	\begin{itemize}
		\item nový paměťový model
		\begin{itemize}
			\item dva výrazy jsou v kolizi pokud přistupují k jedné proměnné a alespoň jeden výraz do ní zapisuje
			\item pokud jsou v programu dvě vlákna, které jsou v kolizi nad ne-atomickou proměnnou a tyto vlákna nemají striktní pořadí, jde o race condition
			\item race condition = nedefinované chování\footnote{demo 15}\footnote{demo 16}\footnote{demo 17}
		\end{itemize}
	\end{itemize}
\end{frame}

\section{Kontaktní informace}
	\subsection{Pracovní}

\begin{frame}[label=kontakt-simontoth]
	\frametitle{Mgr. Šimon Tóth}
	\begin{itemize}
		\item{Gotex (Šumavská 15) - kanceláře Cesnet (B310)}
	\end{itemize}

	\begin{columns}
	\column{0.5\textwidth}
		Fakulta Informatiky (MU)
		\begin{itemize}
			\item{toth@fi.muni.cz}
			\item{tel. 549 49 6446}
		\end{itemize}

	\column{0.5\textwidth}
		Cesnet z.s.p.o.
		\begin{itemize}
			\item{simon@cesnet.cz}
			\item{kancelář C122}
			\item{tel. 234 680 235}
		\end{itemize}

	\end{columns}
\end{frame}

	\subsection{Osobní}

\begin{frame}
	\frametitle{Mgr. Šimon Tóth}
		Osobní
		\begin{itemize}
			\item{kontakt@simontoth.cz}
			\item{tel. 776 565 424}
		\end{itemize}
\end{frame}




\end{document}
