\documentclass[pdftex,aspectratio=169]{beamer}

% Setup beamer look theme
\mode<presentation>
{
	\useinnertheme{rectangles}
	\useoutertheme{infolines}
	\usecolortheme{crane}
	\setbeamertemplate{navigation symbols}{}
}

% Use nicer font
\usepackage{palatino}
\usepackage{palatino,graphics,array}

% Setup UTF-8 encoding
\usepackage[utf8]{inputenc}
\usepackage[T1]{fontenc}

% Setup czech language
\usepackage[czech]{babel}

% Include usefull packages
\usepackage{graphics}

% Include hyperlinks support
\usepackage{hyperref}
\hypersetup{colorlinks=true,linkcolor=black,urlcolor=black,unicode=true}

% Add support for simple URLs
\usepackage{url}

% Commands for CC license
\newcommand{\CcLongnameByNcSa}{\href{http://creativecommons.org/licenses/by-nc-sa/3.0/cz/}{Attribution-Noncommercial-ShareAlike}}
\newcommand{\CcImageBy}[1]{\includegraphics[scale=#1]{../commons/graphics/cc-by-white.pdf}}
\newcommand{\CcImageNc}[1]{\includegraphics[scale=#1]{../commons/graphics/cc-nc-white.pdf}}
\newcommand{\CcImageSa}[1]{\includegraphics[scale=#1]{../commons/graphics/cc-sa-white.pdf}}
\newcommand{\CcGroupByNcSa}[2]{\CcImageBy{#1}\hspace*{#2}\CcImageNc{#1}\hspace*{#2}\CcImageSa{#1}}




\title{GIT}
\author[]{Mgr.~Šimon~Tóth}
\institute[CESNET]{MetaCentrum @ Cesnet}
\date{\today}

\newcommand{\CcNote}[1]{% longname
        Licencováno pod: \textit{Creative Commons #1 3.0 License}.%
}

\usepackage{tikz}
\usetikzlibrary{chains,fit,shapes}

\begin{document}

\pgfdeclarelayer{background}
\pgfdeclarelayer{middle}
\pgfsetlayers{background,middle,main}

\begin{frame}
\titlepage
	\vfill
	\begin{center}
		\CcGroupByNcSa{0.33}{0.95ex}\\ {\tiny\CcNote{\CcLongnameByNcSa}}
		\vspace*{2ex}
	\end{center}
\end{frame}

\section{Úvod}
\subsection*{Prezentace}

\begin{frame}
	\frametitle{Doporučené čtení}
	\begin{itemize}
		\item{\texttt{man git}}
		\item{\url{http://book.git-scm.com}}
		\item{\url{http://progit.org/book}}
	\end{itemize}
\end{frame}

\begin{frame}
	\frametitle{Obsah}
	\tableofcontents
\end{frame}

\subsection{GIT}

\begin{frame}
	\frametitle{Technologie na pozadí}
	\begin{itemize}
		\item{SHA1 identifikátory comitů}
		\begin{itemize}
			\item{deduplication}
			\item{rychlé vyhledávání}
			\item{unikátní identifikace}
			\item{detekce chyb}
		\end{itemize}
		\item{tři typy objektů}
		\begin{description}
			\item[\textbf{blob}]{blok dat}
			\item[\textbf{tree}]{reference na jiné stromy a bloby\footnote{strom nemůže být prázdný}}
			\item[\textbf{commit}]{odkazuje na specifický strom + meta informace}
			\item[\textbf{tag}]{odkazuje na specifický strom}
		\end{description}
	\end{itemize}
\end{frame}

\begin{frame}
	\frametitle{Ukládání dat}
	\begin{itemize}
		\item{každý comit referencuje celý strom}
		\item{silná dependence na deduplikaci pomocí SHA1}
	\end{itemize}
\end{frame}

\subsection{Distribuované VCS}

\begin{frame}
\frametitle{Centralizovaný VCS}

\begin{figure}
\begin{tikzpicture}[auto,swap]
	\node[draw=blue,fill=blue!20,rectangle,rounded corners,label=above:\tiny$checkout$,font=\tiny] (soubor1) at (0,0) {$soubor$};
	\node[draw=blue,fill=blue!20,rectangle,rounded corners,label=above:\tiny$checkout$,font=\tiny] (soubor2) at (0,2) {$soubor$};

	\node[draw=green,fill=green!20,rectangle,rounded corners,font=\tiny] (ver1) at (3,0) {$version_1$};
	\node[draw=green,fill=green!20,rectangle,rounded corners,font=\tiny] (ver2) at (3,1) {$version_2$};
	\node[draw=green,fill=green!20,rectangle,rounded corners,font=\tiny] (ver3) at (3,2) {$version_3$};

	\draw[-] (ver1) -- (ver2);
	\draw[-] (ver2) -- (ver3);

	\draw[->] (ver3) -- (soubor2);
	\draw[->] (ver2) -- (soubor1);

	\begin{pgfonlayer}{middle}
		\node[draw=black,fill=white,fit=(ver1) (ver2) (ver3),label=above:\tiny$version db$] (verdb) {};
	\end{pgfonlayer}

	\begin{pgfonlayer}{background}
		\node[scale=2,draw=black,fill=gray!20,fit=(soubor1),label=below:$PC_1$] (pc1) {};
		\node[scale=2,draw=black,fill=gray!20,fit=(soubor2),label=below:$PC_2$] (pc2) {};
		\node[xscale=2,yscale=1.24,draw=black,fill=gray!20,fit=(ver1) (ver2) (ver3),label=below:$Server$] (svr) {};
	\end{pgfonlayer}
\end{tikzpicture}
\end{figure}

\end{frame}

\begin{frame}
	\frametitle{Decentralizovaný VCS}

\begin{figure}
\begin{tikzpicture}[auto,swap]
	\node[draw=blue,fill=blue!20,rectangle,rounded corners,label=above:\tiny$checkout$,font=\tiny] (soubor1) at (0,3) {$soubor$};
	\node[draw=blue,fill=blue!20,rectangle,rounded corners,label=above:\tiny$checkout$,font=\tiny] (soubor2) at (6,3) {$soubor$};

	\node[draw=green,fill=green!20,rectangle,rounded corners,font=\tiny] (ver11) at (0,0) {$version_1$};
	\node[draw=green,fill=green!20,rectangle,rounded corners,font=\tiny] (ver12) at (0,1) {$version_2$};
	\node[draw=green,fill=green!20,rectangle,rounded corners,font=\tiny] (ver13) at (0,2) {$version_3$};

	\node[draw=green,fill=green!20,rectangle,rounded corners,font=\tiny] (ver21) at (3,0) {$version_1$};
	\node[draw=green,fill=green!20,rectangle,rounded corners,font=\tiny] (ver22) at (3,1) {$version_2$};
	\node[draw=green,fill=green!20,rectangle,rounded corners,font=\tiny] (ver23) at (3,2) {$version_3$};

	\node[draw=green,fill=green!20,rectangle,rounded corners,font=\tiny] (ver31) at (6,0) {$version_1$};
	\node[draw=green,fill=green!20,rectangle,rounded corners,font=\tiny] (ver32) at (6,1) {$version_2$};
	\node[draw=green,fill=green!20,rectangle,rounded corners,font=\tiny] (ver33) at (6,2) {$version_3$};

	\draw[-] (ver11) -- (ver12);	\draw[-] (ver12) -- (ver13);
	\draw[-] (ver21) -- (ver22);	\draw[-] (ver22) -- (ver23);
	\draw[-] (ver31) -- (ver32);	\draw[-] (ver32) -- (ver33);

	\draw[->] (ver13) -- (soubor1);	\draw[->] (ver33) -- (soubor2);


	\begin{pgfonlayer}{middle}
		\node[draw=black,fill=white,fit=(ver11) (ver12) (ver13),label=below:\tiny$version db$] (verdb1) {};
		\node[draw=black,fill=white,fit=(ver21) (ver22) (ver23),label=above:\tiny$version db$] (verdb2) {};
		\node[draw=black,fill=white,fit=(ver31) (ver32) (ver33),label=below:\tiny$version db$] (verdb3) {};
	\end{pgfonlayer}

	\draw[<->,thick] (verdb1) -- (verdb2);	\draw[<->,thick] (verdb2) -- (verdb3);

	\begin{pgfonlayer}{background}
		\node[xscale=2,yscale=1.24,draw=black,fill=gray!20,fit=(soubor1) (verdb1),label=below:$PC_1$] (pc1) {};
		\node[xscale=2,yscale=1.24,draw=black,fill=gray!20,fit=(soubor2) (verdb3),label=below:$PC_2$] (pc2) {};
		\node[xscale=2,yscale=1.44,draw=black,fill=gray!20,fit=(ver1) (ver2) (ver3),label=below:$Server$] (svr) {};
	\end{pgfonlayer}
\end{tikzpicture}
\end{figure}

\end{frame}

\begin{frame}
	\frametitle{Výhody a nevýhody}
	\begin{description}
		\item[$+$]{lokální operace}
		\item[$-$]{náročnost na diskový prostor}
		\item[$\pm$]{nevynucuje specifický workflow}
		\item[$\pm$]{je to jiné}
	\end{description}
\end{frame}

\section{Praktický GIT}
\subsection{Lehký úvod}

\begin{frame}
	\frametitle{Staging area}
	\begin{figure}
		\begin{tikzpicture}[auto,swap]
			\node[draw=green,fill=green!20,rectangle,rounded corners,font=\tiny] (work) at (0,3) {$working$ $directory$};
			\node[draw=orange,fill=orange!20,rectangle,rounded corners,font=\tiny] (stage) at (0,2) {$stage$ $directory$};
			\node[draw=blue,fill=blue!20,rectangle,rounded corners,font=\tiny] (repo) at (0,1) {$repository$};
			\draw[->,thick] (work) .. controls +(180:1.5cm) and +(180:1.5cm) .. (stage) node [midway,right,text centered] {\tiny\texttt{git add}};
			\draw[->,thick] (stage) .. controls +(0:1.5cm) and +(0:1.5cm) .. (repo) node [midway,left,text centered] {\tiny\texttt{git commit}};
		\end{tikzpicture}
	\end{figure}
\end{frame}

\begin{frame}
	\frametitle{Úvodní demonstrace práce s gitem}
Vytvoření nového repozitáře pomocí \texttt{git init}.
Vytvoření souboru a ukázka \texttt{git status}.
Přidání souboru pomocí \texttt{git add}.
Úprava souboru a ukázka rozdílu mezi stage a working directory.
\texttt{git diff}.
\end{frame}

\section{Vlastnosti GITu}
\subsection{Branches}

% Git branches
% Git merging
% Git resolving collisions
% Git fast-forward merging

\begin{frame}
	\frametitle{Branches}
\end{frame}

\subsection{Status}

% Git log
% Git status
% Git diff

\begin{frame}
	\frametitle{Information}
\end{frame}


\subsection{Remotes}

% Git remotes
% Git push, pull, clone
% Git tracking

\begin{frame}
	\frametitle{Remotes}
\end{frame}

% git configurace
% git index, working directory
% clonning, initializing

\subsection{Git konfigurace}

% Git ignore files
% 

\begin{frame}
	\frametitle{Configuration}
\end{frame}


\subsection{Git rebase}

% Git rebase
% Git interactive rebase

\begin{frame}
	\frametitle{Rebase}
\end{frame}

% vetve v gitu, mergovani (asi i pull, push a remotes)
% git vs. classical server sided
% workflow, rebasovani a cistota stromu
% big files

\subsection{Git advanced}

\begin{frame}
	\frametitle{Interactive adding}
\end{frame}

\begin{frame}
	\frametitle{Stashing}
\end{frame}

\begin{frame}
	\frametitle{Commit spec}
\end{frame}

\begin{frame}
	\frametitle{Git grep}
\end{frame}

\subsection{Undo}

\begin{frame}
	\frametitle{Reset, checkout, revert}
\end{frame}

\begin{frame}
	\frametitle{Maintainence}
	% git gc
	% git fsck
\end{frame}

\subsection{Git sharing}

\begin{frame}
	\frametitle{Bare repository}
	% bare repository
	% http sharing
	% git sharing
	% git ssh
	% git hooks
\end{frame}

\begin{frame}
	\frametitle{Creating empty branch}
\end{frame}

\begin{frame}
	\frametitle{Mass editing history}
	% git filter-branch
\end{frame}

\begin{frame}
	\frametitle{Multiway merge}
\end{frame}

\subsection{Subtree and Submodule}

\begin{frame}
	\frametitle{Subtree and Submodule}
\end{frame}

\begin{frame}
	\frametitle{Git bisect}
\end{frame}

\begin{frame}
	\frametitle{Git blame}
\end{frame}

\begin{frame}
	\frametitle{Git patch emailing - jenom okrajove}
\end{frame}

\begin{frame}
	\frametitle{Git recovery}
\end{frame}

\begin{frame}
	\frametitle{Git submodules}
\end{frame}

\begin{frame}
	\frametitle{Git and subversion}
\end{frame}

\end{document}

