\documentclass[pdftex]{beamer}

% Setup UTF-8 encoding
\usepackage[utf8]{inputenc}
\usepackage[T1]{fontenc}

% Setup beamer look theme
\mode<presentation>
{
	\useinnertheme{rectangles}
	\useoutertheme{infolines}
	\usecolortheme{dove}
	\setbeamertemplate{navigation symbols}{}
}

% Use nicer font
\usepackage{palatino,graphics,array,url,tikz,verbatim,hyperref}

% Setup czech language
\usepackage[czech]{babel}

% Include hyperlinks support
\hypersetup{colorlinks=true,linkcolor=black,urlcolor=black,unicode=true}

\usetikzlibrary{arrows,shapes}
\usetikzlibrary{chains,fit}
\usetikzlibrary{mindmap,trees}

\title{GIT hands-on}
\author[]{Mgr.~Šimon~Tóth}
\date{\today}

\begin{document}

\pgfdeclarelayer{background}
\pgfdeclarelayer{middle}
\pgfsetlayers{background,middle,main}

\begin{frame}
	\titlepage
\end{frame}

\section{Úvod}
\subsection*{Obsah přednášky}

\begin{frame}
\frametitle{O čem to nebude}
	\begin{itemize}
		\item	porovnání \texttt{svn}, \texttt{cvs} vs. \texttt{git}
		\item	velmi pokročilé vlastnosti
	\end{itemize}
\end{frame}

\begin{frame}
\frametitle{Doporučené čtení}
	\begin{itemize}
		\item	\texttt{man git}
		\item	\url{http://book.git-scm.com}
		\item	\url{http://progit.org/book}
	\end{itemize}
\end{frame}

\begin{frame}
\frametitle{Obsah}
	\tableofcontents
\end{frame}

\subsection{GIT}

\begin{frame}
\frametitle{Technologie na pozadí}
	\begin{itemize}
		\item	SHA1 identifikátory comitů
		\begin{itemize}
			\item	deduplication
			\item	rychlé vyhledávání
			\item	unikátní identifikace
			\item	detekce chyb
		\end{itemize}

		\item	tři typy objektů
		\begin{description}
			\item[\textbf{blob}]	blok dat
			\item[\textbf{tree}]	reference na jiné stromy a bloby\footnote{strom nemůže být prázdný}
			\item[\textbf{commit}]	odkazuje na specifický strom + meta informace
			\item[\textbf{tag}]	odkazuje na specifický strom
		\end{description}
	\end{itemize}
\end{frame}

\begin{frame}
	\frametitle{Ukládání dat}
	\begin{itemize}
		\item	každý comit referencuje celý strom
		\item	silná dependence na deduplikaci pomocí SHA1
	\end{itemize}
\end{frame}

\subsection{Distribuované VCS}

\begin{frame}
\frametitle{Centralizovaný VCS}

\begin{figure}
\begin{tikzpicture}[auto,swap]
	\node[draw=blue,fill=blue!20,rectangle,rounded corners,label=above:\tiny$checkout$,font=\tiny] (soubor1) at (0,0) {$soubor$};
	\node[draw=blue,fill=blue!20,rectangle,rounded corners,label=above:\tiny$checkout$,font=\tiny] (soubor2) at (0,2) {$soubor$};

	\node[draw=green,fill=green!20,rectangle,rounded corners,font=\tiny] (ver1) at (3,0) {$version_1$};
	\node[draw=green,fill=green!20,rectangle,rounded corners,font=\tiny] (ver2) at (3,1) {$version_2$};
	\node[draw=green,fill=green!20,rectangle,rounded corners,font=\tiny] (ver3) at (3,2) {$version_3$};

	\draw[-] (ver1) -- (ver2);
	\draw[-] (ver2) -- (ver3);

	\draw[->] (ver3) -- (soubor2);
	\draw[->] (ver2) -- (soubor1);

	\begin{pgfonlayer}{middle}
		\node[draw=black,fill=white,fit=(ver1) (ver2) (ver3),label=above:\tiny$version db$] (verdb) {};
	\end{pgfonlayer}

	\begin{pgfonlayer}{background}
		\node[scale=2,draw=black,fill=gray!20,fit=(soubor1),label=below:$PC_1$] (pc1) {};
		\node[scale=2,draw=black,fill=gray!20,fit=(soubor2),label=below:$PC_2$] (pc2) {};
		\node[xscale=2,yscale=1.24,draw=black,fill=gray!20,fit=(ver1) (ver2) (ver3),label=below:$Server$] (svr) {};
	\end{pgfonlayer}
\end{tikzpicture}
\end{figure}

\end{frame}

\begin{frame}
	\frametitle{Decentralizovaný VCS}

\begin{figure}
\begin{tikzpicture}[auto,swap]
	\node[draw=blue,fill=blue!20,rectangle,rounded corners,label=above:\tiny$checkout$,font=\tiny] (soubor1) at (0,3) {$soubor$};
	\node[draw=blue,fill=blue!20,rectangle,rounded corners,label=above:\tiny$checkout$,font=\tiny] (soubor2) at (6,3) {$soubor$};

	\node[draw=green,fill=green!20,rectangle,rounded corners,font=\tiny] (ver11) at (0,0) {$version_1$};
	\node[draw=green,fill=green!20,rectangle,rounded corners,font=\tiny] (ver12) at (0,1) {$version_2$};
	\node[draw=green,fill=green!20,rectangle,rounded corners,font=\tiny] (ver13) at (0,2) {$version_3$};

	\node[draw=green,fill=green!20,rectangle,rounded corners,font=\tiny] (ver21) at (3,0) {$version_1$};
	\node[draw=green,fill=green!20,rectangle,rounded corners,font=\tiny] (ver22) at (3,1) {$version_2$};
	\node[draw=green,fill=green!20,rectangle,rounded corners,font=\tiny] (ver23) at (3,2) {$version_3$};

	\node[draw=green,fill=green!20,rectangle,rounded corners,font=\tiny] (ver31) at (6,0) {$version_1$};
	\node[draw=green,fill=green!20,rectangle,rounded corners,font=\tiny] (ver32) at (6,1) {$version_2$};
	\node[draw=green,fill=green!20,rectangle,rounded corners,font=\tiny] (ver33) at (6,2) {$version_3$};

	\draw[-] (ver11) -- (ver12);	\draw[-] (ver12) -- (ver13);
	\draw[-] (ver21) -- (ver22);	\draw[-] (ver22) -- (ver23);
	\draw[-] (ver31) -- (ver32);	\draw[-] (ver32) -- (ver33);

	\draw[->] (ver13) -- (soubor1);	\draw[->] (ver33) -- (soubor2);


	\begin{pgfonlayer}{middle}
		\node[draw=black,fill=white,fit=(ver11) (ver12) (ver13),label=below:\tiny$version db$] (verdb1) {};
		\node[draw=black,fill=white,fit=(ver21) (ver22) (ver23),label=above:\tiny$version db$] (verdb2) {};
		\node[draw=black,fill=white,fit=(ver31) (ver32) (ver33),label=below:\tiny$version db$] (verdb3) {};
	\end{pgfonlayer}

	\draw[<->,thick] (verdb1) -- (verdb2);	\draw[<->,thick] (verdb2) -- (verdb3);

	\begin{pgfonlayer}{background}
		\node[xscale=2,yscale=1.24,draw=black,fill=gray!20,fit=(soubor1) (verdb1),label=below:$PC_1$] (pc1) {};
		\node[xscale=2,yscale=1.24,draw=black,fill=gray!20,fit=(soubor2) (verdb3),label=below:$PC_2$] (pc2) {};
		\node[xscale=2,yscale=1.44,draw=black,fill=gray!20,fit=(ver1) (ver2) (ver3),label=below:$Server$] (svr) {};
	\end{pgfonlayer}
\end{tikzpicture}
\end{figure}

\end{frame}

\begin{frame}
	\frametitle{Výhody a nevýhody}
	\begin{description}
		\item[$+$]{lokální operace}
		\item[$-$]{náročnost na diskový prostor}
		\item[$\pm$]{nevynucuje specifický workflow}
		\item[$\pm$]{je to jiné}
	\end{description}
\end{frame}

\section{Praktický GIT}
\subsection{Lehký úvod}

\begin{frame}
	\frametitle{Staging area}
	\begin{figure}
		\begin{tikzpicture}[auto,swap]
			\node[draw=green,fill=green!20,rectangle,rounded corners,font=\tiny] (work) at (0,3) {$working$ $directory$};
			\node[draw=orange,fill=orange!20,rectangle,rounded corners,font=\tiny] (stage) at (0,2) {$stage$ $directory$};
			\node[draw=blue,fill=blue!20,rectangle,rounded corners,font=\tiny] (repo) at (0,1) {$repository$};
			\draw[->,thick] (work) .. controls +(180:1.5cm) and +(180:1.5cm) .. (stage) node [midway,right,text centered] {\tiny\texttt{git add}};
			\draw[->,thick] (stage) .. controls +(0:1.5cm) and +(0:1.5cm) .. (repo) node [midway,left,text centered] {\tiny\texttt{git commit}};
		\end{tikzpicture}
	\end{figure}
\end{frame}

\begin{frame}
	\frametitle{Demo č.1}
	\begin{itemize}
		\item{vytvoření repozitáře}
		\item{přidání souboru a commit}
		\item{diff, log, status}
		\item{revert}
		\item{další operace (rm,mv)}
	\end{itemize}
\end{frame}

\section{Větve}
\subsection{Základní práce}

\begin{frame}
	\frametitle{Větvě a git}
	\begin{itemize}
		\item vytvoření větve je velmi levná operace
		\begin{itemize}
			\item lokální operace
		\end{itemize}
		\item přepínání je rozumně rychlé
		\item u mergování je snaha automatizovat
	\end{itemize}
\end{frame}

\begin{frame}
	\frametitle{Demo č.2}
	\begin{itemize}
		\item vytvoření větvě
		\item přepínání
		\item \texttt{git gui}
		\item mergování
		\item fast forward
		\item multiway merge
	\end{itemize}
\end{frame}

\subsection{Rebase}

\begin{frame}
	\frametitle{Rebase}
	\begin{figure}
		\begin{tikzpicture}[auto,swap]
			\node[draw=green,fill=green!20,circle,font=\tiny] (x1) at (0,0) {$x_1$};
			\node[draw=green,fill=green!20,circle,font=\tiny] (x2) at (0,1) {$x_2$};
			\node[draw=green,fill=green!20,circle,font=\tiny] (x3) at (0,2) {$x_3$};

			\node<3->[draw=green,fill=green!20,circle,font=\tiny] (x4) at (0,3) {$x_4$};
			\node<3->[draw=green,fill=green!20,circle,font=\tiny] (x5) at (0,4) {$x_5$};

			\node<2-3>[draw=blue,fill=blue!20,circle,font=\tiny] (y1) at (2,3) {$y_1$};
			\node<2-3>[draw=blue,fill=blue!20,circle,font=\tiny] (y2) at (2,4) {$y_2$};

			\node<4->[draw=blue,fill=blue!20,circle,font=\tiny] (z1) at (0,5) {$y_{1}^{*}$};
			\node<4->[draw=blue,fill=blue!20,circle,font=\tiny] (z2) at (0,6) {$y_{2}^{*}$};

			\draw[-] (x1) -- (x2);		\draw[-] (x2) -- (x3);
			\draw<3->[-] (x3) -- (x4);	\draw<3->[-] (x4) -- (x5);
			\draw<2-3>[-] (x3) -- (y1);	\draw<2-3>[-] (y1) -- (y2);
			\draw<4->[-] (x5) -- (z1);	\draw<4->[-] (z1) -- (z2);
		\end{tikzpicture}
	\end{figure}
\end{frame}

\begin{frame}
	\frametitle{Rebase}
	\begin{itemize}
		\item přepisuje historii
		\item umožňuje zachovat lineární větve
		\item interaktivní rebase pro hluboké změny
	\end{itemize}
\end{frame}

\begin{frame}
	\frametitle{Demo č.3}
	\begin{itemize}
		\item \texttt{rebase}
		\item \texttt{rebase {-}{-}onto}
		\item \texttt{rebase {-}{-}interactive}
	\end{itemize}
\end{frame}

\begin{frame}
	\frametitle{Rebase}
	\begin{itemize}
		\item příprava patche/pull requestu
		\item lokální úklid
		\item dlouhodobý vývoj vůči upstreamu
		\item developer / integrator workflow
	\end{itemize}
\end{frame}

\subsection{Cherry picking}

\begin{frame}
	\frametitle{Cherry pick}
	\begin{itemize}
		\item vyzobávání comitů
	\end{itemize}
\end{frame}

\begin{frame}
	\frametitle{Demo č.4}
	\begin{itemize}
		\item cherry picking
	\end{itemize}
\end{frame}

\section{Remotes}
\subsection{Veřejné repozitáře}

\begin{frame}
	\frametitle{Veřejné repozitáře}
	\begin{itemize}
		\item remotes
		\item bare repo
		\item push
		\item hooks
	\end{itemize}
\end{frame}

\subsection{Upstream}

\begin{frame}
	\frametitle{Upstream}
	\begin{itemize}
		\item clone
		\item tracking
		\item fetch \& pull
	\end{itemize}
\end{frame}

\section{Git nástroje}

\subsection{Údržba}

\begin{frame}
	\frametitle{Údržba gitu}
	\begin{itemize}
		\item recovery
		\item git gc
		\item git fsck
	\end{itemize}
\end{frame}

\subsection{Nástroje}

\begin{frame}
	\frametitle{Nástroje}
	\begin{itemize}
		\item git grep
		\item git bisect
		\item git blame
	\end{itemize}
\end{frame}


\section{XXXX}

\subsection{Status}

% Git log
% Git status
% Git diff

\begin{frame}
	\frametitle{Information}
\end{frame}

% Git ignore files

\begin{frame}
	\frametitle{Configuration}
\end{frame}

% workflow
% big files

\subsection{Git advanced}

\begin{frame}
	\frametitle{Stashing}
\end{frame}

\begin{frame}
	\frametitle{Commit spec}
\end{frame}

\subsection{Undo}

\begin{frame}
	\frametitle{Reset, checkout, revert}
\end{frame}


\begin{frame}
	\frametitle{Creating empty branch}
\end{frame}

\begin{frame}
	\frametitle{Mass editing history}
	% git filter-branch
\end{frame}

\subsection{Subtree and Submodule}

\begin{frame}
	\frametitle{Subtree and Submodule}
\end{frame}

\begin{frame}
	\frametitle{Git patch emailing - jenom okrajove}
\end{frame}

\begin{frame}
	\frametitle{Git submodules}
\end{frame}

\begin{frame}
	\frametitle{Git and subversion}
\end{frame}

\end{document}

