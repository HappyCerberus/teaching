\documentclass[10pt]{article}

\usepackage[utf8]{inputenc}
\usepackage[T1]{fontenc}

\begin{document}

\section{Základní práce s repozitářem}
\begin{enumerate}
\item stazeni aktualni upstream verze
\begin{itemize}
	\item \texttt{git clone git://github.com/HappyCerberus/presentation.git}
	\begin{itemize}
		\item vytvoreni lokalni kopie
		\item kopie je kompletni, nastavi se ale pouze jedna vetev
		\item \texttt{git branch -a}
	\end{itemize}
\end{itemize}

\item{provedeme upravy}
\item{prohlednuti aktualnich zmen}
\begin{itemize}
	\item \texttt{git status}
	\item \texttt{git diff}
	\item \texttt{git diff origin/master}
\end{itemize}

\item{vytvoreni commitu}
\begin{itemize}
	\item v gitu se preferují drobné commity
	\item \texttt{git add file1.txt}
\end{itemize}
\end{enumerate}

\texttt{git add soubor}
\texttt{git status}
- ukazat jak vypada status ted

\texttt{git diff}
- ukazka diffu

\texttt{uprava}
- ukazat co se stane kdyz upravim soubor

\texttt{git commit}
- upozornit na commit message

\texttt{git gui}
- podivat se na commit tree

\end{document}
