\documentclass[pdftex,aspectratio=169]{beamer}

% Setup beamer look theme
\mode<presentation>
{
	\useinnertheme{rectangles}
	\useoutertheme{infolines}
	\usecolortheme{crane}
	\setbeamertemplate{navigation symbols}{}
}

% Use nicer font
\usepackage{palatino}
\usepackage{palatino,graphics,array}

% Setup UTF-8 encoding
\usepackage[utf8]{inputenc}
\usepackage[T1]{fontenc}

% Setup czech language
\usepackage[czech]{babel}

% Include usefull packages
\usepackage{graphics}

% Include hyperlinks support
\usepackage{hyperref}
\hypersetup{colorlinks=true,linkcolor=black,urlcolor=black,unicode=true}

% Add support for simple URLs
\usepackage{url}

% Commands for CC license
\newcommand{\CcLongnameByNcSa}{\href{http://creativecommons.org/licenses/by-nc-sa/3.0/cz/}{Attribution-Noncommercial-ShareAlike}}
\newcommand{\CcImageBy}[1]{\includegraphics[scale=#1]{../commons/graphics/cc-by-white.pdf}}
\newcommand{\CcImageNc}[1]{\includegraphics[scale=#1]{../commons/graphics/cc-nc-white.pdf}}
\newcommand{\CcImageSa}[1]{\includegraphics[scale=#1]{../commons/graphics/cc-sa-white.pdf}}
\newcommand{\CcGroupByNcSa}[2]{\CcImageBy{#1}\hspace*{#2}\CcImageNc{#1}\hspace*{#2}\CcImageSa{#1}}




\title{Úvod do jazyka C}
\subtitle{Cvičení - týden V}
\author[]{Mgr.~Šimon~Tóth}
\institute[FI@MU]{Fakulta informatiky @ Masarykova Univerzita}
\date{\today}

\newcommand{\CcNote}[1]{% longname
        Licencováno pod: \textit{Creative Commons #1 3.0 License}.%
}

\begin{document}
	\begin{frame}
		\titlepage
		\vfill
		\begin{center}
			\CcGroupByNcSa{0.33}{0.95ex}\\
			{\tiny\CcNote{\CcLongnameByNcSa}}
			\vspace*{2ex}
		\end{center}
	\end{frame}

\section{Organizační info}
\subsection{Úvodní test}

\begin{frame}
	\frametitle{Úvodní test}
	\begin{itemize}
		\item{\texttt{Student -> Odpovědníky}}
	\end{itemize}
\end{frame}

\subsection{Kritické období}

\begin{frame}
	\frametitle{Kritické období}
	\begin{itemize}
		\item{tohle je velmi kritické období}
		\item{nepodceňte písemku}
		\item{pozor na aktuálně probíranou látku}
	\end{itemize}
\end{frame}

\section{Teorie}
\subsection{Důležité oblasti}

\begin{frame}
	\frametitle{(Strict) aliasing}
	\begin{itemize}
		\item{různé datové typy mají různé požadavky na adresu}
		\item{dva ukazatele různých typů nesmí ukazovat na stejnou adresu}
		\item{povolené varianty:}
		\begin{itemize}
			\item{změna znaménka}
			\item{změna kvalifikátoru}
			\item{na union, který obsahuje daný typ jako jednu z položek}
			\item{mezi položkami jednoho unionu (rozporuplná definice v standardu)}
		\end{itemize}
	\end{itemize}
\end{frame}

\begin{frame}
	\frametitle{Dynamická alokace}
	\begin{itemize}
		\item{bacha na ni}
	\end{itemize}
\end{frame}

\begin{frame}
	\frametitle{Valgrind}
	\begin{itemize}
		\item{vždy si program otestujte i přes valgrind}
		\item{pokud vám valgrind píše chyby, neodevzdávejte!}
	\end{itemize}
\end{frame}

\section{Práce na cvičení}
\subsection{Bludiště}

\begin{frame}
	\frametitle{Bludiště}
	\begin{itemize}
		\item{na vstupu máte bludiště}
		\begin{itemize}
			\item{\texttt{"\space"} - cesta}
			\item{\texttt{"\#"} - stěna}
			\item{\texttt{"S"} - start}
			\item{\texttt{"C"} - cíl}
		\end{itemize}
		\item{každý řádek je stejně dlouhý}
		\item{bludiště je uzavřené}
		\item{na okraji ale nemusí být stěny}
		\item{bludiště nemá předem dané rozměry}
	\end{itemize}
\end{frame}

\begin{frame}
	\frametitle{Požadavky}
	\begin{itemize}
		\item{jediným omezením musí být volná paměť}
		\item{bludiště musí být uloženo v dvourozměrném poli}
		\begin{itemize}
			\item{ideálně po řádcích}
		\end{itemize}
		\item{po načtení bludiště pro kontrolu opište}
		\item{pokud budete mít hotovo, naprogramujte hledání cesty}
		\item{\url{https://gist.github.com/2134019}}
	\end{itemize}
\end{frame}

\section{Kontaktní informace}
	\subsection{Pracovní}

\begin{frame}[label=kontakt-simontoth]
	\frametitle{Mgr. Šimon Tóth}
	\begin{itemize}
		\item{Gotex (Šumavská 15) - kanceláře Cesnet (B310)}
	\end{itemize}

	\begin{columns}
	\column{0.5\textwidth}
		Fakulta Informatiky (MU)
		\begin{itemize}
			\item{toth@fi.muni.cz}
			\item{tel. 549 49 6446}
		\end{itemize}

	\column{0.5\textwidth}
		Cesnet z.s.p.o.
		\begin{itemize}
			\item{simon@cesnet.cz}
			\item{kancelář C122}
			\item{tel. 234 680 235}
		\end{itemize}

	\end{columns}
\end{frame}

	\subsection{Osobní}

\begin{frame}
	\frametitle{Mgr. Šimon Tóth}
		Osobní
		\begin{itemize}
			\item{kontakt@simontoth.cz}
			\item{tel. 776 565 424}
		\end{itemize}
\end{frame}




\end{document}
