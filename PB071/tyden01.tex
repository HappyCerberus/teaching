\documentclass[pdftex,aspectratio=169]{beamer}

% Setup beamer look theme
\mode<presentation>
{
	\useinnertheme{rectangles}
	\useoutertheme{infolines}
	\usecolortheme{crane}
	\setbeamertemplate{navigation symbols}{}
}

% Use nicer font
\usepackage{palatino}
\usepackage{palatino,graphics,array}

% Setup UTF-8 encoding
\usepackage[utf8]{inputenc}
\usepackage[T1]{fontenc}

% Setup czech language
\usepackage[czech]{babel}

% Include usefull packages
\usepackage{graphics}

% Include hyperlinks support
\usepackage{hyperref}
\hypersetup{colorlinks=true,linkcolor=black,urlcolor=black,unicode=true}

% Add support for simple URLs
\usepackage{url}

% Commands for CC license
\newcommand{\CcLongnameByNcSa}{\href{http://creativecommons.org/licenses/by-nc-sa/3.0/cz/}{Attribution-Noncommercial-ShareAlike}}
\newcommand{\CcImageBy}[1]{\includegraphics[scale=#1]{../commons/graphics/cc-by-white.pdf}}
\newcommand{\CcImageNc}[1]{\includegraphics[scale=#1]{../commons/graphics/cc-nc-white.pdf}}
\newcommand{\CcImageSa}[1]{\includegraphics[scale=#1]{../commons/graphics/cc-sa-white.pdf}}
\newcommand{\CcGroupByNcSa}[2]{\CcImageBy{#1}\hspace*{#2}\CcImageNc{#1}\hspace*{#2}\CcImageSa{#1}}




\title{Úvod do jazyka C}
\subtitle{Cvičení - týden I}
\author[]{Mgr.~Šimon~Tóth}
\institute[FI@MU]{Fakulta informatiky @ Masarykova Univerzita}
\date{\today}

\newcommand{\CcNote}[1]{% longname
        Licencováno pod: \textit{Creative Commons #1 3.0 License}.%
}

\begin{document}
	\begin{frame}
		\titlepage
		\vfill
		\begin{center}
			\CcGroupByNcSa{0.33}{0.95ex}\\
			{\tiny\CcNote{\CcLongnameByNcSa}}
			\vspace*{2ex}
		\end{center}
	\end{frame}

\section{Organizační info}
\subsection{Představení}
        
\begin{frame}
	\frametitle{Představení}
	\begin{itemize}
		\item{Mgr. Šimon Tóth}
		\begin{itemize}
			\item{externí lektor C a C++ na FI}
			\item{výzkumný pracovník Cesnet}
			\item{Ph.D. student}
			\item{kontaktní info na konci slidů}
		\end{itemize}
	\end{itemize}
\end{frame}

\subsection{Organizační informace}

\begin{frame}
	\frametitle{Vstupní požadavky}
	\begin{itemize}
		\item{práce s UNIXem, LINUXem (lze dohnat v průběhu cvičení)}
		\item{znalost algoritmizace na úrovni předmětu IB001}
		\begin{itemize}
			\item{schopnost přepisu slovního popisu algoritmu do zdrojového kódu}
			\item{znalost základních programovacích konceptů}
			\begin{itemize}
				\item{proměnné}
				\item{podmínky}
				\item{cykly}
				\item{funkce/podprogramy}
			\end{itemize}
			\item{nepředpokládá se znalost syntaxe jazyka C}
		\end{itemize}
	\end{itemize}
\end{frame}

\begin{frame}
	\frametitle{Obecně platné požadavky}
	\begin{itemize}
		\item{účast je povinná (max. 2 absence)}
		\begin{itemize}
			\item{v případě překročení o 1 jde řešit náhradním příkladem}
		\end{itemize}
		\item{týdenní příklady jsou pro vás povinné}
		\begin{itemize}
			\item{vzhledem k minimu 0 se ale nekontroluje odevzdávání}
			\item{pouze součet bodů}
		\end{itemize}
		\item{přednášky jsou silně doporučované}
		\begin{itemize}
			\item{budou průběžné mini-testy}
		\end{itemize}
	\end{itemize}
\end{frame}

\subsection{Materiály}

\begin{frame}
	\frametitle{Výukové materiály}
	\begin{itemize}
		\item{\url{http://elearning.simontoth.cz}}
		\item{\url{http://stackoverflow.com/}}
		\item{\url{http://www.linuxmanpages.com/}}
		\item{\url{http://cplusplus.com} (C++)}
		\item{\url{http://www.cppreference.com/wiki/} (C++)}
	\end{itemize}
\end{frame}

\begin{frame}
	\frametitle{Dostupnost slidů}
	\begin{itemize}
		\item{\url{http://elearning.simontoth.cz}}
		\item{\url{https://github.com/HappyCerberus/teaching}}
	\end{itemize}
\end{frame}

\section{Nástroje}
\subsection{Linux}

\begin{frame}
	\frametitle{Linux}
	\begin{itemize}
		\item{domácí úkoly se odevzdávají na Linuxový server (aisa)}
		\item{práce na cvičení bude probíhat na Linuxových strojích}
	\end{itemize}
	\begin{block}{Silně doporučuji!}
		\begin{itemize}
		\item{nainstalovat si doma Linux}
		\item{vybrat si jeden zkonzolový editor}
		\item{vybrat si jeden grafický editor/IDE}
		\end{itemize}
	\end{block}
\end{frame}

\subsection{Provoz Linuxu}

\begin{frame}
	\frametitle{Provoz Linuxu}
	\begin{itemize}
		\item{dual boot}
		\item{virtualizace}
		\item{semi-virtualizovaná řešení}
	\end{itemize}
\end{frame}

\begin{frame}
	\frametitle{Cygwin}
	\href{http://www.cygwin.com/}{http://www.cygwin.com/}
	\begin{description}
		\item[+]{jednoduché na rozchození}
		\item[-]{\alert{stará verze GCC}}
		\item[-]{nedoporučuji používat}
	\end{description}
\end{frame}

\begin{frame}
	\frametitle{Dual boot}
	\begin{description}
		\item[+]{plný výkon}
		\item[+]{plnohodnotný Linux}
		\item[-]{nutné rebooty}
	\end{description}
\end{frame}

\begin{frame}
	\frametitle{Plná emulace}
	\begin{description}
		\item[+]{jednoduché na rozchození}
		\item[+]{plnohodnotný Linux}
		\item[$\pm$]{náročné na výkon}
	\end{description}
\end{frame}

\begin{frame}
	\frametitle{coLinux}
	\href{http://www.colinux.org/}{http://www.colinux.org/}
	\begin{description}
		\item[+]{jednoduché na rozchození}
		\item[+]{dobrý výkon}
		\item[-]{pouze příkazová řádka\footnote{je možné rozjet i GUI}}
	\end{description}
\end{frame}

\subsection{Konzolové Editory}

\begin{frame}
	\frametitle{Konzolové Editory}
	\begin{itemize}
		\item{naučte se rozumně ovládat alespoň jeden konzolový editor}
		\item{někomu může konzolový editor naprosto stačit}
	\end{itemize}
\end{frame}

\begin{frame}
	\frametitle{VIM}
	\begin{itemize}
		\item{jednoduše rozšiřitelný}
		\item{spousta funkcí v základu}
		\item{pomocí pluginů předčí leckteré IDE}
	\end{itemize}
\end{frame}

\begin{frame}
	\frametitle{Emacs}
	\begin{itemize}
		\item{umí snad všechno}
		\item{nepoužívám}
	\end{itemize}
\end{frame}

\begin{frame}
	\frametitle{Joe,Pico,Nano}
	\begin{itemize}
		\item{jednoduché editory}
		\item{měli by umět zvýraznit syntax}
	\end{itemize}
\end{frame}

\subsection{Grafické Editory a IDE}

\begin{frame}
	\frametitle{Grafické Editory a IDE}
	\begin{itemize}
		\item{pokud nemáte chuť se učit Vim/Emacs}
		\item{IDE poskytují slušný komfort při vývoji}
	\end{itemize}
\end{frame}

\begin{frame}
	\frametitle{Kate}
	\begin{itemize}
		\item{syntax}
		\item{seznam souborů}
		\item{konzole}
	\end{itemize}
\end{frame}

\begin{frame}
	\frametitle{Gedit}
	\begin{itemize}
		\item{podobné funkce jako Kate}
		\item{nepoužívám}
	\end{itemize}
\end{frame}

\begin{frame}
	\frametitle{Netbeans}
	\href{http://www.netbeans.org/}{http://www.netbeans.org/}
	\begin{itemize}
		\item{od verze 5.5 podpora C/C++}
		\item{podpora pro vzdálený vývoj přes ssh}
		\item{nepoužívám}
	\end{itemize}
\end{frame}

\begin{frame}
	\frametitle{Eclipse CDT}
	\href{http://www.eclipse.org/cdt/}{http://www.eclipse.org/cdt/}
	\begin{itemize}
		\item{asi nejlepší IDE pro C}
		\item{problematická instalace ve Windows}
	\end{itemize}
\end{frame}

\begin{frame}
	\frametitle{Cod::Blocks}
	\href{http://www.codeblocks.org/}{http://www.codeblocks.org/}
	\begin{itemize}
		\item{oficiální IDE pro cvičení}
	\end{itemize}
\end{frame}

\section{Kompilace programu}
\subsection{Důležité parametry GCC}

\begin{frame}
	\frametitle{Kategorie parametrů GCC}
	\begin{itemize}
		\item{jazyk}
		\item{debugovací a profilovací informace}
		\item{warningy}
		\item{optimalizace}
	\end{itemize}
\end{frame}

\begin{frame}
	\frametitle{jazyk}
	\begin{description}
		\item[-ansi]{kompilace podle normy ISO C90 \alert{NEPOUŽÍVAT}}
		\item[-std=c99]{kompilace podle normy ISO C99}
		\item[-pedantic]{striktní dodržování normy}
	\end{description}
\end{frame}

\begin{frame}
	\frametitle{warování}
	\begin{description}
		\item[-Wall]{zapne hlavní sadu varování\footnote{\alert{nutné}}}
		\item[-Wextra]{zapne další důležitá varování\footnote{silně doporučuji}} 
	\end{description}
\end{frame}

\begin{frame}
	\frametitle{debugovací a profilovací info.}
	\begin{description}
		\item[-g(1-3)]{přidání degubovacích informací do binárky}
		\item[-ggdb(1-3)]{přidání debugovacích informací do binárky pro program gdb (používejte)}
		\item[-p]{přidání profilovacích informací do binárky}
		\item[-pg]{přidání profilovacích informací do binárky pro program gprof}
	\end{description}
\end{frame}

\begin{frame}
	\frametitle{optimalizace}
	\begin{description}
		\item[-O(0-3)]{zapnutí úrovně optimalizace}
		\item[-Os]{optimalizace velikosti binárky}
		\item[-s]{strip, odstranění všech symbolů z binárky\footnote[1]{\alert{znemožní debugování}}}
		\item[-fomit-frame-pointer]{odstranění frame pointeru\footnotemark[1]}
	\end{description}
\end{frame}

\subsection{Clang}

\begin{frame}
	\frametitle{Clang}
	\begin{itemize}
		\item{moderní kompilátor C a C++}
		\item{prozatím menší sada statických kontrol (warningů)}
		\item{poskytuje ale velmi čitelné chybové výstupy}
		\item{pokračující práce na statickém analyzátoru}
	\end{itemize}
\end{frame}

\section{Práce na cvičení}
\subsection{Vstupní test}

\begin{frame}
	\frametitle{Vstupní test}
	\begin{itemize}
		\item{IS -> Student -> Odpovědníky -> C vstupní test}
	\end{itemize}
\end{frame}

\subsection{Standardní obsah cvičení}

\begin{frame}
	\frametitle{Práce na cvičení}
	\begin{itemize}
		\item{\href{http://elearning.simontoth.cz/public/pb071\_cviceni}{http://elearning.simontoth.cz/public/pb071\_cviceni}}
	\end{itemize}
\end{frame}

\section{Kontaktní informace}
	\subsection{Pracovní}

\begin{frame}[label=kontakt-simontoth]
	\frametitle{Mgr. Šimon Tóth}
	\begin{itemize}
		\item{Gotex (Šumavská 15) - kanceláře Cesnet (B310)}
	\end{itemize}

	\begin{columns}
	\column{0.5\textwidth}
		Fakulta Informatiky (MU)
		\begin{itemize}
			\item{toth@fi.muni.cz}
			\item{tel. 549 49 6446}
		\end{itemize}

	\column{0.5\textwidth}
		Cesnet z.s.p.o.
		\begin{itemize}
			\item{simon@cesnet.cz}
			\item{kancelář C122}
			\item{tel. 234 680 235}
		\end{itemize}

	\end{columns}
\end{frame}

	\subsection{Osobní}

\begin{frame}
	\frametitle{Mgr. Šimon Tóth}
		Osobní
		\begin{itemize}
			\item{kontakt@simontoth.cz}
			\item{tel. 776 565 424}
		\end{itemize}
\end{frame}




\end{document}
