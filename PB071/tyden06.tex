\documentclass[pdftex,aspectratio=169]{beamer}

% Setup beamer look theme
\mode<presentation>
{
	\useinnertheme{rectangles}
	\useoutertheme{infolines}
	\usecolortheme{crane}
	\setbeamertemplate{navigation symbols}{}
}

% Use nicer font
\usepackage{palatino}
\usepackage{palatino,graphics,array}

% Setup UTF-8 encoding
\usepackage[utf8]{inputenc}
\usepackage[T1]{fontenc}

% Setup czech language
\usepackage[czech]{babel}

% Include usefull packages
\usepackage{graphics}

% Include hyperlinks support
\usepackage{hyperref}
\hypersetup{colorlinks=true,linkcolor=black,urlcolor=black,unicode=true}

% Add support for simple URLs
\usepackage{url}

% Commands for CC license
\newcommand{\CcLongnameByNcSa}{\href{http://creativecommons.org/licenses/by-nc-sa/3.0/cz/}{Attribution-Noncommercial-ShareAlike}}
\newcommand{\CcImageBy}[1]{\includegraphics[scale=#1]{../commons/graphics/cc-by-white.pdf}}
\newcommand{\CcImageNc}[1]{\includegraphics[scale=#1]{../commons/graphics/cc-nc-white.pdf}}
\newcommand{\CcImageSa}[1]{\includegraphics[scale=#1]{../commons/graphics/cc-sa-white.pdf}}
\newcommand{\CcGroupByNcSa}[2]{\CcImageBy{#1}\hspace*{#2}\CcImageNc{#1}\hspace*{#2}\CcImageSa{#1}}




\title{Úvod do jazyka C}
\subtitle{Cvičení - týden \texttt{VI}}
\author[]{Mgr.~Šimon~Tóth}
\institute[FI@MU]{Fakulta informatiky @ Masarykova Univerzita}
\date{\today}

\newcommand{\CcNote}[1]{% longname
        Licencováno pod: \textit{Creative Commons #1 3.0 License}.%
}

\begin{document}
	\begin{frame}
		\titlepage
		\vfill
		\begin{center}
			\CcGroupByNcSa{0.33}{0.95ex}\\
			{\tiny\CcNote{\CcLongnameByNcSa}}
			\vspace*{2ex}
		\end{center}
	\end{frame}

\section{Opakování}
\subsection{Bludiště}

\begin{frame}
	\frametitle{Bludiště}
	\begin{itemize}
		\item{vzorové řešení bludiště}
	\end{itemize}
\end{frame}

\section{Organizační info}
\subsection{Kritické období}

\begin{frame}
	\frametitle{Kritické období}
	\begin{itemize}
		\item{tohle je velmi kritické období}
		\item{pozor na aktuálně probíranou látku}
	\end{itemize}
\end{frame}

\section{Teorie}
\subsection{Struktura}
\begin{frame}
	\frametitle{Struktura}
	\begin{itemize}
		\item{agregovaný datový typ}
		\item{jednotlivé položky v paměti za sebou}
		\item{položky jsou zarovnané podle svých potřeb}\pause
		\item{struktury mají svůj vlastní namespace}
		\begin{itemize}
			\item \texttt{struct osoba x;}
			\item \texttt{osoba x;}
		\end{itemize}
		\item{pro zjednodušení můžeme použít \texttt{typedef}}
	\end{itemize}
\end{frame}

\subsection{Union}
\begin{frame}
	\frametitle{Union}
	\begin{itemize}
		\item{agregovaný datový typ}
		\item{jednotlivé položky v paměti přes sebe}\pause
		\item{uniony mají svůj vlastní namespace}
		\begin{itemize}
			\item \texttt{union cislo x;}
			\item \texttt{cislo x;}
		\end{itemize}
		\item{pro zjednodušení můžeme použít \texttt{typedef}}
	\end{itemize}
\end{frame}

\section{Práce na cvičení}
\subsection{Strom}

\begin{frame}
	\frametitle{Strom}
	\begin{itemize}
		\item{na vstupu máte strom v závorkové notaci}
		\item{tento strom rozparsujte a načtěte do vhodné datové struktury}
	\end{itemize}
\end{frame}

\section{Kontaktní informace}
	\subsection{Pracovní}

\begin{frame}[label=kontakt-simontoth]
	\frametitle{Mgr. Šimon Tóth}
	\begin{itemize}
		\item{Gotex (Šumavská 15) - kanceláře Cesnet (B310)}
	\end{itemize}

	\begin{columns}
	\column{0.5\textwidth}
		Fakulta Informatiky (MU)
		\begin{itemize}
			\item{toth@fi.muni.cz}
			\item{tel. 549 49 6446}
		\end{itemize}

	\column{0.5\textwidth}
		Cesnet z.s.p.o.
		\begin{itemize}
			\item{simon@cesnet.cz}
			\item{kancelář C122}
			\item{tel. 234 680 235}
		\end{itemize}

	\end{columns}
\end{frame}

	\subsection{Osobní}

\begin{frame}
	\frametitle{Mgr. Šimon Tóth}
		Osobní
		\begin{itemize}
			\item{kontakt@simontoth.cz}
			\item{tel. 776 565 424}
		\end{itemize}
\end{frame}




\end{document}
